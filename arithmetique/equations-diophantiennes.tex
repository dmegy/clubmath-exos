\documentclass[11pt,a4paper]{article}
\usepackage[french]{babel}
\usepackage[utf8]{inputenc}
\usepackage{mathtools,amssymb,amsthm}
\usepackage{mathrsfs,stmaryrd}
\usepackage{fancybox,mdframed,multicol,comment,enumitem}
\usepackage{microtype}

\usepackage{hyperref}
\hypersetup{
    colorlinks=true,       % false: boxed links; true: colored links
    linkcolor=blue,          % color of internal links
    citecolor=blue,        % color of links to bibliography
    filecolor=blue,      % color of file links
    urlcolor=black           % color of external links
}

%\usepackage[dvipsnames]{xcolor}

\usepackage[normalem]{ulem} % pour souligner avec changements de ligne
\usepackage{pgf,pgfmath,tikz}
\usetikzlibrary{arrows}
\usetikzlibrary[patterns]
\tikzset{every picture/.style={execute at begin picture={
   \shorthandoff{:;!?};}
}}

\newcommand*\circled[1]{\tikz[baseline=(char.base)]{
            \node[shape=circle,draw,inner sep=2pt] (char) {#1};}}


\def\point{node {$\bullet$}}

\usepackage{tkz-euclide}



\theoremstyle{definition}
\newtheorem{theoreme}{Théorème}[section]
\newtheorem{definition}[theoreme]{Définition}
\newtheorem{definitions}[theoreme]{Définitions}
\newtheorem{lemme}[theoreme]{Lemme}
\newtheorem{proposition}[theoreme]{Proposition}
\newtheorem{corollaire}[theoreme]{Corollaire}
\newtheorem{remarque}[theoreme]{Remarque}
\newtheorem{ex}{Problème}

\newcommand{\N}{\mathbb N}
\newcommand{\Z}{\mathbb Z}
\newcommand{\Q}{\mathbb Q}
\newcommand{\R}{\mathbb R}
\newcommand{\C}{\mathbb C}
\newcommand{\U}{\mathbb U}
\newcommand{\F}{\mathbb F}
\newcommand{\G}{\mathbb G}

\newcommand*{\etoile}
{
\begin{center}
$\star$\par
$\star$\hspace*{3ex}$\star$
\end{center}
}

\newcommand{\ensemble}[2]{\left \{ #1  
    \ifx&#2&%
       %
    \else%
       \, \middle | \, #2%
    \fi%
\right \}}

\newcommand{\modulo}[1]{\:\left(\operatorname{mod}\:#1\right)}

% délimiteurs

\DeclarePairedDelimiter{\abs}{\lvert}{\rvert}
\DeclarePairedDelimiter{\ceil}{\lceil}{\rceil}
\DeclarePairedDelimiter{\floor}{\lfloor}{\rfloor}


%%%%%%%%%%%%%%%%%%%%%%%%%%%%%%%%%
%%%%%% MISE EN FORME CLUB %%%%%%%
%%%%%%%%%%%%%%%%%%%%%%%%%%%%%%%%%

\pagestyle{empty}

\usepackage[margin=2.5cm]{geometry}
\everymath{\displaystyle}
\usepackage{fourier}
%\usepackage{fourier,eulervm}% Adobe Utopia et Euler
% En-tête des feuilles :

\newcommand{\enTete}[1]{
\noindent \textbf{\textsf{\href{http://depmath-nancy.univ-lorraine.fr/club/}{Club Mathématique de Nancy} \hfill Institut Élie Cartan}}
\hrule
\begin{center}
{\Huge \textbf{#1}}
\end{center}
\hrule
\vspace{1em}
}

\newcommand{\avertissement}{\begin{mdframed}[linewidth=1pt]\textbf{AVERTISSEMENT ! Ce document est un brouillon qui sert de catalogue pour les feuilles d'exos du club mathématique de Nancy \url{https://dmegy.perso.math.cnrs.fr/club/}. Ne pas diffuser tel quel aux élèves ni de façon large sur le net, il reste des coquilles et énoncés parfois peu précis. Ce document a vocation a rester inachevé. Il peut néanmoins être utile aux enseignants. Enfin, ce document change en permanence, la version à jour est récupérable sur \url{https://github.com/dmegy/clubmath-exos}.}\end{mdframed}}




% - - - - - - - - - - - - - -
% PARAMETRAGE DU PACKAGE ANSWERS 
% POUR LES INDICATIONS ET CORRECTIONS
% - - - - - - - - - - - - - - 

\usepackage{answers}

\Newassociation{sol}{Soln}{solutions}
% ira dans le fichier d'identifiant 'solutions'
% et écrira les solutions dans un environnement 'Soln'
\Newassociation{hint}{Hint}{indications}

\newenvironment{exo}{\begin{ex} \label{enonce.\theex} }{\end{ex} }

\renewenvironment{Soln}[1]{\noindent{\bf Correction de l'exercice \ref{enonce.#1}.} \\ }

\renewenvironment{Hint}[1]{ \noindent{\bf Exercice  \ref{enonce.#1}.} \label{hint.#1}}


% - - - - - - - - - - - - - - 
% FIN PARAMETRAGE ANSWERS
% - - - - - - - - - - - - - - 

%-----------------------------
% MACROS POUR LES FEUILLES DE TD


\newenvironment{feuilleTD}{


\Opensolutionfile{indications}[\jobname_hints]
\Opensolutionfile{solutions}[\jobname_sol]
}{
\Closesolutionfile{indications}
\Closesolutionfile{solutions}
}

\newcommand{\indications}{
\newpage
\noindent {\Large \bf Indications} \hrulefill

\vspace{1em}
\Readsolutionfile{indications}
}

\newcommand{\correction}{
\newpage
\hrule
\begin{center}
{\Large \bf Correction}
\end{center}
\hrule
\vspace{1em}
\Readsolutionfile{solutions}
}




\begin{document}
\Opensolutionfile{indications}[_\jobname_hints]
\Opensolutionfile{solutions}[_\jobname_sol]


\title{Équations diophantiennes}
\author{Damien Mégy}
\maketitle
\avertissement
\section{Équations linéaires}


\begin{exo}[Non testé]
Résoudre sur $\Z$ :  (ou sur $\N$ ?)
\[ \begin{cases}
8x+5y+z&=100\\
x+y+z&=20
\end{cases}\]
\begin{hint}
\end{hint}
\begin{sol}
Source : \url{https://www.youtube.com/watch?v=bd960eS5GAs}
On peut virer les $z$ trop facilement...

Attention on résout sur $\N$ ?

La vidéo présente ça avec des congruences, ok.
\end{sol}
\end{exo}

\section{Équations de degré supérieur}

(Séparer suivant la signature ? )

\begin{exo}
Factoriser $x^4-3x^2+1$ comme produit de deux polynômes de degré deux \emph{à coefficients entiers}.
\begin{hint}
Attention si on fait le changement de variable tout de suite, on tombe sur des radicaux. Il faut alors regrouper les racines.

Ou alors faire $x^4-2x^2+1 -x^2$ et factoriser comme différence de carrés.
\end{hint}
\begin{sol}
\end{sol}
\end{exo}

\begin{exo}[Bouger dans arithmétique. bof]
Soient $a$ et $b$ des entiers relatifs. On suppose que $a+ab+b=52$.
Quelles sont les valeurs possibles pour $a+b$?
\begin{hint}
Factoriser. Attention aux nombres négatifs.
\end{hint}
\begin{sol}
$52$ ou $-56$.
\end{sol}
\end{exo}


\begin{exo}
Résoudre $x^2+3x+9=9y^2$ sur $\Z$.
\begin{hint}
\end{hint}
\begin{sol}
Source : \url{https://www.youtube.com/watch?v=BdhrFqYzd-w}
Pas mal du tout. Technique utilisée : trinôme en $x$, dire que le discriminant doit être un carré parfait, le nommer, injecter, refactoriser...
\end{sol}
\end{exo}



\begin{exo}
Résoudre sur $\N$ l'équation 
\[ \sqrt x + \sqrt y = \sqrt{250}\]
\begin{hint}
Séparer $x$ et $y$. Montrer que $10x$ doit être un carré parfait, puis que $x$ doit être de la forme $10k^2$, puis injecter.
\end{hint}
\begin{sol}
Noter que $x$ et $y$ sont $\leq 250$ donc en théorie on pourrait \og brute-forcer\fg, mais en pratique ça serait un peu long.

Après avoir suivi l'indication, on se retrouve avec une majoration $|k|\leq 5$ donc là on pteste tout.

Autre méthode : si $sqty x + \sqty y = 5\sqrt{10}$, on a forcément $x$ et $y$ de la forme $a\sqrt{10}$ ? 


Source : \url{https://www.youtube.com/watch?v=cqyjsjSi-2c}
\end{sol}
\end{exo}

\begin{exo}
Trouver les solutions entières de l'équation
\[ x^2-xy+y^2 = x+y.\]
\begin{hint}
\end{hint}
\begin{sol}
Isoler les $x$ d'un côté, trinôme en $x$, le dscriminant doit être défini et entier.

Autre méthode : tout ramener d'un côté, multiplier par deux, somme de carrés !!

On trouve
\[ (x-y)^2 + (x-1)^2+(y-1)^2=2\]

Finir avec soin, il y a plein de cas.

Autre technique :
\[ (2x - y  - 1)^2 = -(3y^2 - 6y - 1)\]
Donc $-3y^2-6y-1$ doit être négative, mais ça redonne la même chose que la technique avec le disciminant.
\end{sol}
\end{exo}

\begin{exo}[Nontesté]
Résoudre sur $\Z$ : 
\[ \begin{cases}
x^2-y^2-z^2 &=1\\
y+z-x&=3
\end{cases}\]
\begin{hint}
\end{hint}
\begin{sol}
Source : \url{https://www.youtube.com/watch?v=vmZNecBOIKw}

\end{sol}
\end{exo}


\begin{exo}
\begin{hint}
\end{hint}
\begin{sol}
\end{sol}
\end{exo}


\begin{exo}
\begin{hint}
\end{hint}
\begin{sol}
\end{sol}
\end{exo}




\begin{exo}
Résoudre $x^2+y^2=z^2$ c'est-à-dire paramétrer les triplets pythagoriciens.
\begin{hint}
\end{hint}
\begin{sol}
\end{sol}
\end{exo}


\begin{exo}
Résoudre $\frac1x+\frac1y = \frac1z$ c'est-à-dire paramétrer les  solutions. Exhiber plusieurs solutions obtenues de cette façon.
\begin{hint}
L'équation est homogène : factoriser le pgcd.
\end{hint}
\begin{sol}
Source : \url{https://www.youtube.com/watch?v=f7vKUzGYev4}
\end{sol}
\end{exo}


\begin{exo}
Déterminer tous les couples de nombres premiers $(x,y)$ qui vérifient l'équation suivante : 
\[ x^2-2y^2 = 1\]
\begin{hint}
\end{hint}
\begin{sol}
Source : \url{https://www.youtube.com/watch?v=Wl1NCS9OySw} attention il ne cherche que les solutions avec des nombres premiers.
Mettre la résolution complète ?
\end{sol}
\end{exo}



\begin{exo}
\begin{hint}
\end{hint}
\begin{sol}
\end{sol}
\end{exo}


\Closesolutionfile{indications}
\Closesolutionfile{solutions}

\indications
\correction

\end{document}

