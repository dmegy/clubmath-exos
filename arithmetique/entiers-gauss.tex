\documentclass[11pt,a4paper]{article}
\usepackage[french]{babel}
\usepackage[utf8]{inputenc}
\usepackage{mathtools,amssymb,amsthm}
\usepackage{mathrsfs,stmaryrd}
\usepackage{fancybox,mdframed,multicol,comment,enumitem}
\usepackage{microtype}

\usepackage{hyperref}
\hypersetup{
    colorlinks=true,       % false: boxed links; true: colored links
    linkcolor=blue,          % color of internal links
    citecolor=blue,        % color of links to bibliography
    filecolor=blue,      % color of file links
    urlcolor=black           % color of external links
}

%\usepackage[dvipsnames]{xcolor}

\usepackage[normalem]{ulem} % pour souligner avec changements de ligne
\usepackage{pgf,pgfmath,tikz}
\usetikzlibrary{arrows}
\usetikzlibrary[patterns]
\tikzset{every picture/.style={execute at begin picture={
   \shorthandoff{:;!?};}
}}

\newcommand*\circled[1]{\tikz[baseline=(char.base)]{
            \node[shape=circle,draw,inner sep=2pt] (char) {#1};}}


\def\point{node {$\bullet$}}

\usepackage{tkz-euclide}



\theoremstyle{definition}
\newtheorem{theoreme}{Théorème}[section]
\newtheorem{definition}[theoreme]{Définition}
\newtheorem{definitions}[theoreme]{Définitions}
\newtheorem{lemme}[theoreme]{Lemme}
\newtheorem{proposition}[theoreme]{Proposition}
\newtheorem{corollaire}[theoreme]{Corollaire}
\newtheorem{remarque}[theoreme]{Remarque}
\newtheorem{ex}{Problème}

\newcommand{\N}{\mathbb N}
\newcommand{\Z}{\mathbb Z}
\newcommand{\Q}{\mathbb Q}
\newcommand{\R}{\mathbb R}
\newcommand{\C}{\mathbb C}
\newcommand{\U}{\mathbb U}
\newcommand{\F}{\mathbb F}
\newcommand{\G}{\mathbb G}

\newcommand*{\etoile}
{
\begin{center}
$\star$\par
$\star$\hspace*{3ex}$\star$
\end{center}
}

\newcommand{\ensemble}[2]{\left \{ #1  
    \ifx&#2&%
       %
    \else%
       \, \middle | \, #2%
    \fi%
\right \}}

\newcommand{\modulo}[1]{\:\left(\operatorname{mod}\:#1\right)}

% délimiteurs

\DeclarePairedDelimiter{\abs}{\lvert}{\rvert}
\DeclarePairedDelimiter{\ceil}{\lceil}{\rceil}
\DeclarePairedDelimiter{\floor}{\lfloor}{\rfloor}


%%%%%%%%%%%%%%%%%%%%%%%%%%%%%%%%%
%%%%%% MISE EN FORME CLUB %%%%%%%
%%%%%%%%%%%%%%%%%%%%%%%%%%%%%%%%%

\pagestyle{empty}

\usepackage[margin=2.5cm]{geometry}
\everymath{\displaystyle}
\usepackage{fourier}
%\usepackage{fourier,eulervm}% Adobe Utopia et Euler
% En-tête des feuilles :

\newcommand{\enTete}[1]{
\noindent \textbf{\textsf{\href{http://depmath-nancy.univ-lorraine.fr/club/}{Club Mathématique de Nancy} \hfill Institut Élie Cartan}}
\hrule
\begin{center}
{\Huge \textbf{#1}}
\end{center}
\hrule
\vspace{1em}
}

\newcommand{\avertissement}{\begin{mdframed}[linewidth=1pt]\textbf{AVERTISSEMENT ! Ce document est un brouillon qui sert de catalogue pour les feuilles d'exos du club mathématique de Nancy \url{https://dmegy.perso.math.cnrs.fr/club/}. Ne pas diffuser tel quel aux élèves ni de façon large sur le net, il reste des coquilles et énoncés parfois peu précis. Ce document a vocation a rester inachevé. Il peut néanmoins être utile aux enseignants. Enfin, ce document change en permanence, la version à jour est récupérable sur \url{https://github.com/dmegy/clubmath-exos}.}\end{mdframed}}




% - - - - - - - - - - - - - -
% PARAMETRAGE DU PACKAGE ANSWERS 
% POUR LES INDICATIONS ET CORRECTIONS
% - - - - - - - - - - - - - - 

\usepackage{answers}

\Newassociation{sol}{Soln}{solutions}
% ira dans le fichier d'identifiant 'solutions'
% et écrira les solutions dans un environnement 'Soln'
\Newassociation{hint}{Hint}{indications}

\newenvironment{exo}{\begin{ex} \label{enonce.\theex} }{\end{ex} }

\renewenvironment{Soln}[1]{\noindent{\bf Correction de l'exercice \ref{enonce.#1}.} \\ }

\renewenvironment{Hint}[1]{ \noindent{\bf Exercice  \ref{enonce.#1}.} \label{hint.#1}}


% - - - - - - - - - - - - - - 
% FIN PARAMETRAGE ANSWERS
% - - - - - - - - - - - - - - 

%-----------------------------
% MACROS POUR LES FEUILLES DE TD


\newenvironment{feuilleTD}{


\Opensolutionfile{indications}[\jobname_hints]
\Opensolutionfile{solutions}[\jobname_sol]
}{
\Closesolutionfile{indications}
\Closesolutionfile{solutions}
}

\newcommand{\indications}{
\newpage
\noindent {\Large \bf Indications} \hrulefill

\vspace{1em}
\Readsolutionfile{indications}
}

\newcommand{\correction}{
\newpage
\hrule
\begin{center}
{\Large \bf Correction}
\end{center}
\hrule
\vspace{1em}
\Readsolutionfile{solutions}
}




\begin{document}
\Opensolutionfile{indications}[_\jobname_hints]
\Opensolutionfile{solutions}[_\jobname_sol]


\title{Entiers de Gauss : $\Z[i]$}
\author{Damien Mégy}
\maketitle
%\avertissement
\tableofcontents

\section{Rappels rapides sur $\Z$}
\begin{enumerate}
\item $\Z$ est ce qu'on appelle un anneau commutatif : il y a une addition et une multiplication, qui vérifient les propriétés attendues (distributivité, commutativité c'est-à-dire $ab=ba$, je ne fais pas la liste complète).
\item Dans $\Z$, on dit que $a$ divise $b$ s'il existe $k\in \Z$ tel que $b=ka$. Par exemple $-2$ divise $6$. 
\item Les \emph{unités} sont les éléments inversibles dans $\Z$. Ce sont $1$ et $-1$ (aucun autre nombre relatif n'est inversible \underline{dans $\Z$} (être inversible dans $\Z$ c'est posséder un inverse qui est lui aussi dans $\Z$). Deux éléments qui se déduisent l'un de l'autre par multiplication par un inversible sont dits \emph{associés}. Dans le cas de $\Z$, $n$ est associé à $-n$. Pour les entiers de Gauss, tout ceci sera moins simple.
\item Un élément est dit irréductible si, modulo multiplication par un inversible, il ne possède que deux diviseurs : $1$ et lui-même. Par exemple, à proprement parler, $-3$ possède quatre diviseurs dans $\Z$ : $\pm 1$ et $\pm 3$. Mais justement, si on néglige la multiplication par un inversible, il n'y a que deux diviseurs. Traditionnellement on appelle nombre premier un irréductible positif. Tout nombre entier relatif peut être écrit comme le produit d'irréductibles. Quitte à groupe les éventuels signes, on peut dire que tout nombre relatif est le produit de nombres premiers et d'un inversible. Là encore, avec les entiers de Gauss il y a plus d'inversibles donc il faudra faire attention.
\item On a une division euclidienne : pour tout couple $(a,b)$ d'entiers avec $b$ non nul, on peut diviser $a$ par $b$ avec reste, c'est-à-dire qu'il existe un couple $(q,r)$ avec $|r|<|b|$ (inégalité stricte) tel que $a=bq+r$. On dit que l'anneau est \emph{euclidien}. De là on déduit la plupart des propriétés de $\Z$, dont : 
\item L'anneau $\Z$ est un anneau \emph{factoriel} : la décomposition en produit d'irréductibles est unique modulo permutation des termes et modulo multiplication par des inversibles. Une façon moins alambiquée de le dire est qu'un nombre positif est produit  de nombres premiers de manière unique modulo permutations.
\item L'anneau $\Z$ est un anneau \emph{principal} : en particulier, si on prend deux éléments, ils possèdent un pgcd (unique modulo multiplication par des inversibles, c'est-à-dire dans notre cas, unique modulo le signe). Mais on pourrait dire que $6$ et $4$ ont en fait deux PGCD : $2$ et $-2$, qui sont associés. Dans les entiers de Gauss, là aussi on aura plusieurs pgcd, tous associés entre eux, mais ça sera moins simple d'en choisir un spécial (ici, on choisit le positif). Il y a aussi un ppcm (ou \emph{des} ppcms, tous associés). Et on a des théorèmes sur les pgcd, comme $a\wedge b = a\wedge (a+b)$ ou des choses comme ça. Ceci est la base pour faire tourner l'algorithme d'Euclide et trouver les pgcd en pratique.
\item Deux éléments sont premiers entre eux si leur pgcd est $1$.
\item Et enfin, si on fixe $n$, on peut calculer modulo $n$, avec des congruences. Comme on travaille avec des congruences modulo des entiers, on peut multiplier les congruences \footnote{La multiplication de congruences et quelque chose de miraculeux et profond et surtout, faux pour des congruences générales, par exemple pour des congruences modulo $2\pi$. En effet on a $\pi\equiv 3\pi~[2\pi]$ et $\pi\equiv 5\pi~[2\pi]$ mais on n'a absolument pas $\pi^2 \equiv 15\pi^2~[2\pi]$!!}. Ça aussi, on pourra faire dans $\Z[i]$ mais ça sera un peu différent.
\end{enumerate}


\section{Entiers de Gauss : $\Z[i]$}

Il existe énormément d'anneaux : anneaux de nombres (quadratiques, algébriques), de polynômes (à une ou plusieurs variables), anneaux plus étonnants , par exemple les entiers $p$-adiques (des nombres entiers écrits en base $p$, mais où on autorise une infinité de chiffres!). Tous ne sont pas euclidiens, ni factoriels, ni principaux etc. L'étude des anneaux en général est faite après le bac, c'est le début de la théorie algébrique des nombres et de la géométrie algébrique, et c'est magnifique ! Ici on ne regarde qu'un exemple très spécial d'anneau : l'anneau $\Z[i]$ des entiers de Gauss. Il est extrêmement spécial car comme $\Z$, il est euclidien :
\begin{enumerate}
\item Définition : $\Z[i]$ est l'ensemble des nombres complexes de la forme $a+ib$ avec $a$ et $b$ entiers relatifs. Par exemple $2+3i$, $1$, $2$, $5$, $4i$, $7i$, $5-7i$ etc. Les éléments de $\Z[i]$ sont appelés \emph{entiers de Gauss}.
\item On peut additionner et multiplier comme les nombres complexes. On obtient un anneau commutatif.
\item Nouveauté : on dispose de la conjugaison et du module ! Le module va un peu ressembler à la valeur absolue sur $\Z$, mais la conjugaison est vraiment quelque chose de nouveau que l'on va exploiter.
\item Notation : si $\alpha=a+ib$ est un entier de Gauss, on note $N(\alpha) = |\alpha|^2=\alpha\bar\alpha = a^2+b^2$. On l'appelle la \emph{norme} de $\alpha$. Montrer que la norme est multiplicative : $N(\alpha\beta) = N(\alpha)N(\beta)$. La norme est un entier positif, c'est une des façons standard de repasser des entiers de Gauss aux entiers relatifs usuels. Par exemple si l'entier de Gauss $\alpha$ divise l'entier de Gauss $\beta$, alors l'entier usuel $N(\alpha)$ divise l'entier $N(\beta)$.
\item La définition de divisibilité est la même que dans $\Z$, mais en remplaçant $\Z$ par $\Z[i]$ : si $\alpha$ et $\beta$ sont deux entiers de Gauss, on dit que $\alpha$ divise $\beta$ s'il existe un entier \underline{de Gauss} $\gamma$ tel que $\beta=\alpha\gamma$. Par exemple, $1+i$ divise $2$, car $2 = (1+i)(1-i)$. On voit en passant que $1-i$ divise également $2$. On commence à voir des choses intéressantes.
\item Mais auparavant, les inversibles : comme dans $\Z$, les inversibles sont les entiers de Gauss qui possèdent un inverse qui est un entier de Gauss. Par exemple, $1$ et $-1$ sont inversibles bien sûr. Mais aussi $i$, puisque $\frac{1}{i}=-i$ est un entier de Gauss. Exercice : les entiers de Gauss inversibles sont $\pm 1$ et $\pm i$. Il y en a donc quatre. ceci va un peu compliquer les choses, mais aussi les embellir. 
\item Deux entiers de Gauss qui se déduisent l'un de l'autre par multiplication par un inversible sont dits associés. Donc par exemple $2$ et $-2$ sont associés. Mais ils sont également associés à $2i$ et $-2i$.  Les associés vont donc par paquets de quatre (à part zéro). Noter que $1+2i$ et $-2+i$ sont associés. Voir l'association est un peu moins immédiat, il faut faire attention.
\item Il y a les analogues de nombres premiers : on dit qu'un entier de Gauss est irréductible si, modulo inversible, il  possède exactement deux diviseurs : $1$ et lui-même. ATTENTION : $2$ n'est pas irréductible dans les entiers de Gauss, comme on l'a vu plus haut : $2=(1+i)(1-i)$ et $5$ non plus car $5=(2+i)(2-i)$. Par contre, $3$ est toujours irréductible. Donc attention, les nombres premiers dans $\Z$ ne sont plus forcément \og premiers\fg. dans $\Z[i]$. Un entier de Gauss irréductible est appelé \emph{premier de Gauss}.  De la même façon que $1$ n'est pas premier, les entiers de Gauss $\pm 1$ et $\pm i$ ne sont pas des premiers de Gauss.
\item Attention il existe des premiers de Gauss qui ne sont pas des nombres premiers. Par exemple $1+i$ (et donc ses associés : $-1+i$, $-1-i$ et $1-i$). (Exercice)
\item Il existe une division euclidienne, le module joue le rôle de la valeur absolue. Essayer de deviner comment marche cette division euclidienne : comment diviser $5+3i$ par $2$, par exemple ? Attention, le quotient et le reste ne sont pas uniques, cette fois, mais ce n'est pas très gênant.
\item Les conséquences sont les mêmes : existence de pgcd et unicité à inversible près, existence et unicité de la décomposition en irréductibles !! Tous les lemmes classiques restent grosso-modo les mêmes.
\item Deux éléments sont premiers entre eux si $1$ est un de leurs pgcds.
\end{enumerate}



%Cours ici : https://personal.math.ubc.ca/~anstee/math444/GaussianIntegersfinal.pdf

% et ici :
% https://www.maths.nottingham.ac.uk/plp/pmzcw/download/fnt_chap5.pdf


\begin{exo}
Montrer que si deux entiers relatifs sont premiers entre eux dans $\Z$, alors ils sont premiers entre eux dans $\Z[i]$.
\end{exo}


\begin{exo}
Montrer que $1+2i$, $1-2i$, $7$, $2+3i$ sont des premiers de Gauss.
\end{exo}


\begin{exo}
Parmi les dix premiers nombres premiers, lesquels sont des premiers de Gauss ? Que peut-on conjecturer ? Tester la conjecture sur les nombres nombres premiers suivants.
\end{exo}

\begin{exo}
Trouver tous les entiers de Gauss qui divisent $2$, $3$, $4$, $5$, pousser jusqu'à $10$. En déduire \og la\fg{} factorisation en produit de premiers de Gauss de ces nombres. (Guillemets car la décomposition est unique uniquement modulo inversibles.)
\end{exo}

\begin{exo}
Parmi les entiers de Gauss $a+ib$ avec $1\leq a,b\leq 10$ trouver tous les premiers de Gauss. Faire comme lorsqu'on cherche les nombres premiers entre $1$ et $100$ (ou $1000$) : crible. (Mais avec des multiples complexes. Utiliser la norme pour borner la \og taille\fg{} des diviseurs possibles.)
% https://fr.wikipedia.org/wiki/Entier_de_Gauss#/media/Fichier:Gaussian-primes.svg
\end{exo}

\begin{exo}
Que dire la norme de ces éléments irréductibles ? Que peut-on conjecturer ?
\end{exo}

%https://perso.eleves.ens-rennes.fr/people/antoine.diez/gaussdeuxcarres.pdf

% cours bien :
% https://webusers.imj-prg.fr/~benjamin.girard/LM220_Trois_derniers_cours.pdf


% DS très complet, applications numériques, sommes de carrés preuve constructive:
% https://cahier-de-prepa.fr/mpsi3-fermat/download?id=56

% PERRIN:

% https://www.imo.universite-paris-saclay.fr/~daniel.perrin/TER/anneauxd%27entiers.pdf

% https://www.imo.universite-paris-saclay.fr/~daniel.perrin/Conferences/OlympiadeDP.pdf

% https://www.imo.universite-paris-saclay.fr/~daniel.perrin/Conferences/Bachet.pdf

\begin{exo}
Trouver un exemple de couple d'entiers de Gauss $(a,b)$ pour lequel la division euclidienne de $a$ par $b$ n'est pas unique. (En général, il y a au maximum quatre choix pour $b$ : pourquoi ?)
\end{exo}

% COURS DE CALDERO : http://math.univ-lyon1.fr/~caldero/Z[i].pdf

\begin{exo}
Montrer que si un nombre premier n'est pas irréductible dans $\Z[i]$, alors il est somme de deux carrés.
\end{exo}

\begin{exo}
Montrer que si un ombre premier $p$ est somme de deux carrés, il n'est pas irréductible dans $\Z[i]$ et que cette décomposition en somme de deux carrés est unique.
\end{exo}

\begin{exo}
On suppose que le nombre premier $p$ est congru à $1$ modulo $4$.
Montrer qu'il n'est pas irréductible à l'aide du théorème de Wilson.
\end{exo}

\begin{exo}
Soit $p$ un nombre premier. Montrer que si $p\equiv 3 [4]$, alors il est irréductible dans $\Z[i]$.
\end{exo}

% http://cpge.byethost5.com/pdf/entiers%20de%20Gauss.pdf?i=1

\begin{exo}
Trouver une condition nécessaire et suffisante sur un entier naturel $n$ pour qu'il soit somme de deux carrés.
\end{exo}


\begin{exo}
Quel est le (ou les) pgcd de $11+7i$ et $18-i$ ?
%https://math.stackexchange.com/questions/82350/how-to-calculate-gcd-of-gaussian-integers
\end{exo}

%Autres calculs de pgcd : 
%https://math.stackexchange.com/questions/1969019/how-can-i-find-gcd167i-10-5i-by-using-the-euclidean-algorithm

% et aussi :
%https://math.stackexchange.com/questions/869514/finding-the-gcd-of-two-gaussian-integers

%https://math.stackexchange.com/questions/1204122/find-the-gcd-of-gaussian-integers


\begin{exo}
Montrer (avec des entiers de Gauss, mais trouver également d'autres preuves) que si $n>3$ et $k$ sont des entiers, alors $n!$ ne peut pas être égal à $k^2+1$.
% source : https://math.stackexchange.com/questions/1231443/proving-x21-neq-n-using-gaussian-integer
\end{exo}

\begin{exo}
Trouver tous les triplets pythagoriciens $x^2+y^2=z^2$ lorsque $x\wedge y=1$ grâce aux entiers de Gauss. (C'est-à-dire, trouver une façon de paramétrer tous les triplets pythagoriciens.)
%https://www.isinj.com/mt-usamo/An%20Introduction%20to%20Diophantine%20Equations%20-%20A%20Problem-Based%20Approach%20-%20Andreescu,%20Andrica%20and%20Cucurezeanu%20(Birk,%202011).pdf
\end{exo}

\begin{exo}
Résoudre sur $\Z$ l'équation
\[ x^2+4=y^3.\]
Indication : montrer que $x+2i$ et $x-2i$ doivent être des cubes dans $\Z[i]$. Écrire $x+2i=(a+ib)^3$ et continuer...
% source : https://cp4space.hatsya.com/2013/11/25/crash-course-in-gaussian-integers/
\end{exo}

\begin{exo}
Résoudre sur $\Z$ :
\[ ab+cd = 34, \quad ac-bd=19\]
\begin{hint}
Identités de ??? Celles que l'on obtient avec les nombres complexes.
\end{hint}
\begin{sol}
Avec des entiers de Gauss : 
\[ (a+id)(c+ib) = 19+34i \]
Il s'agit donc de factoriser cet entier de Gauss.

% With other words, the problem is to factor 19+34i as a complex integer. The norm of 19+34i is 1517=37*41 (which also featured in the video!). 37 and 41 are primes which are equal to 1 (mod 4). It is known that you can write these as the sum of two squares and 37=(6+i)(6-i) and 41=(4+5i)(4-5i) are the prime factorizations of 37 and 41 in the complex integers. Finally, 19+34i = (6+i)(4+5i) and from this you get the solutions for a, b, c, d. (+ related factorizations like 19+34i = (-1+6i)(5-4i) etc.)

Sinon, il va s'agir d'écrire des entiers comme somme de deux carrés...

Source : \url{https://www.youtube.com/watch?v=gUh4dkfQ1pU}
\end{sol}
\end{exo}


Une partie des exos de \url{www.fen.bilkent.edu.tr/~franz/nt/mid2a.pdf} et les autres dans les questions d'agreg !

\begin{exo}
\begin{hint}
\end{hint}
\begin{sol}
\end{sol}
\end{exo}


\begin{exo}
\begin{hint}
\end{hint}
\begin{sol}
\end{sol}
\end{exo}




\begin{exo}
\begin{hint}
\end{hint}
\begin{sol}
\end{sol}
\end{exo}



\begin{exo}
\begin{hint}
\end{hint}
\begin{sol}
\end{sol}
\end{exo}



\begin{exo}
\begin{hint}
\end{hint}
\begin{sol}
\end{sol}
\end{exo}


\Closesolutionfile{indications}
\Closesolutionfile{solutions}

\indications
\correction

\end{document}

