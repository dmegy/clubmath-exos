\documentclass[11pt,a4paper]{article}
\usepackage[french]{babel}
\usepackage[utf8]{inputenc}
\usepackage{mathtools,amssymb,amsthm}
\usepackage{mathrsfs,stmaryrd}
\usepackage{fancybox,mdframed,multicol,comment,enumitem}
\usepackage{microtype}

\usepackage{hyperref}
\hypersetup{
    colorlinks=true,       % false: boxed links; true: colored links
    linkcolor=blue,          % color of internal links
    citecolor=blue,        % color of links to bibliography
    filecolor=blue,      % color of file links
    urlcolor=black           % color of external links
}

%\usepackage[dvipsnames]{xcolor}

\usepackage[normalem]{ulem} % pour souligner avec changements de ligne
\usepackage{pgf,pgfmath,tikz}
\usetikzlibrary{arrows}
\usetikzlibrary[patterns]
\tikzset{every picture/.style={execute at begin picture={
   \shorthandoff{:;!?};}
}}

\newcommand*\circled[1]{\tikz[baseline=(char.base)]{
            \node[shape=circle,draw,inner sep=2pt] (char) {#1};}}


\def\point{node {$\bullet$}}

\usepackage{tkz-euclide}



\theoremstyle{definition}
\newtheorem{theoreme}{Théorème}[section]
\newtheorem{definition}[theoreme]{Définition}
\newtheorem{definitions}[theoreme]{Définitions}
\newtheorem{lemme}[theoreme]{Lemme}
\newtheorem{proposition}[theoreme]{Proposition}
\newtheorem{corollaire}[theoreme]{Corollaire}
\newtheorem{remarque}[theoreme]{Remarque}
\newtheorem{ex}{Problème}

\newcommand{\N}{\mathbb N}
\newcommand{\Z}{\mathbb Z}
\newcommand{\Q}{\mathbb Q}
\newcommand{\R}{\mathbb R}
\newcommand{\C}{\mathbb C}
\newcommand{\U}{\mathbb U}
\newcommand{\F}{\mathbb F}
\newcommand{\G}{\mathbb G}

\newcommand*{\etoile}
{
\begin{center}
$\star$\par
$\star$\hspace*{3ex}$\star$
\end{center}
}

\newcommand{\ensemble}[2]{\left \{ #1  
    \ifx&#2&%
       %
    \else%
       \, \middle | \, #2%
    \fi%
\right \}}

\newcommand{\modulo}[1]{\:\left(\operatorname{mod}\:#1\right)}

% délimiteurs

\DeclarePairedDelimiter{\abs}{\lvert}{\rvert}
\DeclarePairedDelimiter{\ceil}{\lceil}{\rceil}
\DeclarePairedDelimiter{\floor}{\lfloor}{\rfloor}


%%%%%%%%%%%%%%%%%%%%%%%%%%%%%%%%%
%%%%%% MISE EN FORME CLUB %%%%%%%
%%%%%%%%%%%%%%%%%%%%%%%%%%%%%%%%%

\pagestyle{empty}

\usepackage[margin=2.5cm]{geometry}
\everymath{\displaystyle}
\usepackage{fourier}
%\usepackage{fourier,eulervm}% Adobe Utopia et Euler
% En-tête des feuilles :

\newcommand{\enTete}[1]{
\noindent \textbf{\textsf{\href{http://depmath-nancy.univ-lorraine.fr/club/}{Club Mathématique de Nancy} \hfill Institut Élie Cartan}}
\hrule
\begin{center}
{\Huge \textbf{#1}}
\end{center}
\hrule
\vspace{1em}
}

\newcommand{\avertissement}{\begin{mdframed}[linewidth=1pt]\textbf{AVERTISSEMENT ! Ce document est un brouillon qui sert de catalogue pour les feuilles d'exos du club mathématique de Nancy \url{https://dmegy.perso.math.cnrs.fr/club/}. Ne pas diffuser tel quel aux élèves ni de façon large sur le net, il reste des coquilles et énoncés parfois peu précis. Ce document a vocation a rester inachevé. Il peut néanmoins être utile aux enseignants. Enfin, ce document change en permanence, la version à jour est récupérable sur \url{https://github.com/dmegy/clubmath-exos}.}\end{mdframed}}




% - - - - - - - - - - - - - -
% PARAMETRAGE DU PACKAGE ANSWERS 
% POUR LES INDICATIONS ET CORRECTIONS
% - - - - - - - - - - - - - - 

\usepackage{answers}

\Newassociation{sol}{Soln}{solutions}
% ira dans le fichier d'identifiant 'solutions'
% et écrira les solutions dans un environnement 'Soln'
\Newassociation{hint}{Hint}{indications}

\newenvironment{exo}{\begin{ex} \label{enonce.\theex} }{\end{ex} }

\renewenvironment{Soln}[1]{\noindent{\bf Correction de l'exercice \ref{enonce.#1}.} \\ }

\renewenvironment{Hint}[1]{ \noindent{\bf Exercice  \ref{enonce.#1}.} \label{hint.#1}}


% - - - - - - - - - - - - - - 
% FIN PARAMETRAGE ANSWERS
% - - - - - - - - - - - - - - 

%-----------------------------
% MACROS POUR LES FEUILLES DE TD


\newenvironment{feuilleTD}{


\Opensolutionfile{indications}[\jobname_hints]
\Opensolutionfile{solutions}[\jobname_sol]
}{
\Closesolutionfile{indications}
\Closesolutionfile{solutions}
}

\newcommand{\indications}{
\newpage
\noindent {\Large \bf Indications} \hrulefill

\vspace{1em}
\Readsolutionfile{indications}
}

\newcommand{\correction}{
\newpage
\hrule
\begin{center}
{\Large \bf Correction}
\end{center}
\hrule
\vspace{1em}
\Readsolutionfile{solutions}
}



\begin{document}
\Opensolutionfile{indications}[_\jobname_hints]
\Opensolutionfile{solutions}[_\jobname_sol]


\title{Autour du théorème de Thalès}
\author{Damien Mégy}
\maketitle

\avertissement 


\tableofcontents

\section{Uniquement avec le théorème des milieux}



% - - - - - - - -
% - - - - - - - -
% - - - - - - - -
\begin{exo}[Trapèze isocèle]%exo7
% tags : symétrie centrale, réflexion orthogonale, théorème des milieux
Soit $ABC$ un triangle,  $P$ le pied de la hauteur issue de $A$ et $I$, $J$, $K$  les milieux des côtés $[AB]$, $[AC]$ et $[BC]$.
Montrer que $IJKP$ est un trapèze isocèle.
\begin{center}
\begin{tikzpicture}[scale=.7]
\draw[very thick] (0,6) node[above] {$A$}
-- (-2,0) node[left] {$B$}
-- (8,0) node[right] {$C$}
-- cycle;
\draw[dashed,thick] (0,6) -- (0,0) ;
\draw (0,0) rectangle (.6,.6);

\draw[thick] 
(0,0) \point node[below] {$P$}
-- (-1,3) \point node[left] {$I$}
-- (4,3) \point node[right] {$J$}
-- (3,0) \point node[below] {$K$};
\end{tikzpicture}
\end{center}

\begin{sol}

D'après le théorème des milieux, les droites $(IJ)$ et $(BC)$ sont parallèles, donc $IJKP$ est un trapèze.
Il reste à montrer que $IP=JK$.

Toujours d'après le théorème des milieux, on a $AB=2JK$ et donc :
\[ JK=AI=IB.\]

D'autre part, dans le triangle $APB$ rectangle en $P$, le point $I$ est le milieu de l'hypoténuse et donc on a $IA=IB=IP$.

Finalement, on obtient donc:
\[ IP=IB=KJ.\]



\end{sol}
\end{exo} 



% - - - - - - - -
% - - - - - - - -
% - - - - - - - -
\begin{exo}[Théorème de Varignon] \label{Varignon}
Soit $ABCD$ un quadrilatère convexe quelconque et $I$, $J$, $K$, $L$ les milieux de ses côtés.
Montrer que $IJKL$ est ... un parallélogramme. (Toujours!)
%Montrer ensuite que l'aire de $ABCD$ est le double de celle de $IJKL$.
\begin{center}
\begin{tikzpicture}
\tkzDefPoints{-4/-1/A, 4/-1.5/B, 3/1.5/C, -2/1/D, 0/0/O}
\tkzDefMidPoint(A,B)\tkzGetPoint{I}
\tkzDefMidPoint(B,C)\tkzGetPoint{J}
\tkzDefMidPoint(C,D)\tkzGetPoint{K}
\tkzDefMidPoint(D,A)\tkzGetPoint{L}


\tkzDrawPolygon[very thick](A,B,C,D)
\tkzDrawPolygon[dashed](I,J,K,L)
\tkzDrawPoints(I,J,K,L)
\tkzMarkSegments[mark=|||](A,I I,B)
\tkzMarkSegments[mark=||](B,J J,C)
\tkzMarkSegments[mark=s](C,K K,D)
\tkzMarkSegments[mark=|](D,L L,A)
\tkzAutoLabelPoints[center=O](A,B,C,D,I,J,K,L)
\end{tikzpicture}
\end{center}
\begin{hint} 
Tracer les diagonales $[AC]$ et $[BD]$. % et utiliser Thalès
\end{hint}
% couper le quadrilatère en deux triangles et montrer que les côtés sont parallèles à l'aide de Thalès.
% pour une deuxième preuve de l'aire, voir 
% http://serge.mehl.free.fr/anx/th_varignon.html
\begin{sol}
\begin{enumerate}
\item 
Dans le triangle $ABC$, en notant $I$ est le milieu de $[AB]$ et $J$ le milieu de $[BC]$, le théorème de Thalès dit que $(IJ)$ est parall\`ele \`a $(AC)$ et $IJ = \frac{1}{2} AC$. On raisonne pareillement avec le triangle $ACD$, ce qui donne $(KL)$ parall\`ele \`a $AC$ et $KL = \frac{1}{2} AC$. Or, un quadrilat\`ere qui a deux c\^ot\'es parall\`eles et de m\^eme longueur est un parall\'elogramme.

\item La preuve la plus élémentaire utilise uniquement qu'une médiane d'un triangle donné le partage en deux triangles de même aire.

Soit $O$ le point d'intersection des diagonales du quadrilat\`ere $ABCD$. On consid\`ere le triangle $AOB$. Soit $O_1$ le point d'intersection de la diagonale $[AC]$ avec $[IL]$ et soit $O_2$ le point d'intersection de $[IJ]$ avec la diagonale $[BD]$.

% Par construction, le quadrilat\`ere $IO_1OO_2$ est un parall\'elogramme. 
 
Par le théorème de Thalès,  $O_1$ est le milieu de $[AO]$ et $O_2$ le milieu de $[BO]$. Les triangles $IO_1A$ et $IO_1B$ ont même aire, de même que les triangles $I0_2O$ et $IO_2B$.

La somme des aires des triangles $AIO_1$ et $IO_2B$ est donc exactement \'egale \`a l'aire du parallélogramme $IO_1OO_2$. 
 
On applique le même raisonnement aux triangles $BCO$, $CDO$ et $ADO$, ce qui signifie que, dans le quadrilatère $ABCD$, la partie compl\'ementaire de $IJKL$ a une aire qui est exactement \'egale \`a celle de $IJKL$, ce qui permet de conclure. \end{enumerate}

\end{sol}
\end{exo}



\begin{exo}[Centre de gravité d'un quadrilatère]
Soit $ABCD$ un quadrilatère et $I$, $J$, $K$ et $L$ les milieux de ses côtés.
Montrer que le milieu du segment $[IK]$ est aussi le milieu du segment $[JL]$.

(Ce point est appelé le \emph{centre de gravité} des quatre points, ou aussi l'\emph{isobarycentre} des quatre points. Attention, ce n'est pas le centre de gravité du quadrilatère plein ! -> exo sur le centre de gravité du quadrilatère plein ?)
% http://mathafou.free.fr/pbw/pb427.html
\begin{center}
\def\isobarycentre{
\tkzDefPoints{0/0/A, 5/0/B, 4/4/C, 1/3/D}
\tkzDefMidPoint(A,B)\tkzGetPoint{I}
\tkzDefMidPoint(B,C)\tkzGetPoint{J}
\tkzDefMidPoint(C,D)\tkzGetPoint{K}
\tkzDefMidPoint(D,A)\tkzGetPoint{L}
\tkzDefMidPoint(I,K)\tkzGetPoint{O}
\tkzDrawPolygon(A,B,C,D)
\tkzDrawPoints(A,B,C,D,I,J,K,L,O)
\tkzMarkSegments[mark=|||](A,I I,B)
\tkzMarkSegments[mark=||](B,J J,C)
\tkzMarkSegments[mark=|](C,K K,D)
\tkzMarkSegments[mark=s](D,L L,A)
\tkzAutoLabelPoints[center=O](A,B,C,D,I,J,K,L)
}
\begin{tikzpicture}[rotate=10,scale=.7]
\isobarycentre{}
\tkzDrawSegment[dashed](I,K)
\tkzMarkSegments[mark=x](I,O O,K)
\end{tikzpicture}
\hspace{1cm}
\begin{tikzpicture}[rotate=10,scale=.7]
\isobarycentre{}
\tkzDrawSegment[dashed](J,L)
\tkzMarkSegments[mark=o](J,O O,L)
\end{tikzpicture}
\end{center}
\end{exo}



% - - - - - - - -
% - - - - - - - -
\begin{exo}[Moulin à vent et milieux]%vecten
Sur l'extérieur d'un triangle $ABC$, on accole des carrés $ABIJ$, $BCKL$ et $CAMN$.
On note $X$, $Y$ et $Z$ les milieux des segments $[MC]$, $[CB]$ et $[BI]$.
Montrer que le triangle $XYZ$ est rectangle isocèle en $Y$.
\begin{center}
\begin{tikzpicture}[scale=.8,rotate=90]
\tkzDefPoint(190:2){A}
\tkzDefPoint(60:2){B}
\tkzDefPoint(-10:2){C}
\tkzDefPoint(0,0){O}
\tkzDefSquare(A,B) \tkzGetFirstPoint{I}\tkzGetSecondPoint{J}
\tkzDefSquare(B,C) \tkzGetFirstPoint{K}\tkzGetSecondPoint{L}
\tkzDefSquare(C,A) \tkzGetFirstPoint{M}\tkzGetSecondPoint{N}
%\tkzDrawPolygon(A,B,C)
\tkzDrawPolygon(A,B,I,J)
\tkzDrawPolygon(B,C,K,L)
\tkzDrawPolygon(C,A,M,N)
\tkzDrawPolygon[very thick](A,B,C)

\tkzDefMidPoint(M,C)\tkzGetPoint{Y}
\tkzDefMidPoint(C,B)\tkzGetPoint{X}
\tkzDefMidPoint(B,J)\tkzGetPoint{Z}
\tkzDrawPoints(X,Y,Z)
\tkzDrawPolygon[dashed](X,Y,Z)
\tkzAutoLabelPoints[center=O](A,B,C,I,J,K,L,M,N,X,Y,Z)
\end{tikzpicture}
\end{center}
\begin{hint}
Commencer par faire le problème \ref{moulin_a_vent}.
\end{hint}
\end{exo}





% - - - - - - - -
% - - - - - - - -
% - - - - - - - -
\begin{exo}[Partage en trois]
%barycentres
Soit $ABCD$ un parallélogramme, $M$ le milieu de $[AB]$ et $N$ le milieu de $[CD]$. Montrer que les droites $(DM)$ et $(BN)$ coupent la diagonale $[AC]$ en deux points $K$ et $L$ le divisant en trois segments égaux.
\begin{center}
\begin{tikzpicture}[rotate=-5]
\tkzDefPoints{-3/-1/A, 2/-1/B, 3/1/C, -2/1/D, 0/0/O}
\tkzDefMidPoint(A,B)\tkzGetPoint{M}
\tkzDefMidPoint(C,D)\tkzGetPoint{N}
\tkzInterLL(D,M)(A,C)\tkzGetPoint{K}
\tkzInterLL(B,N)(A,C)\tkzGetPoint{L}
\tkzDrawPolygon(A,B,C,D)
\tkzDrawSegments[dashed](D,M B,N A,C)
\tkzMarkSegments[mark=||](A,M M,B C,N N,D)
\tkzLabelPoints[above=3pt](D,C,K,N)
\tkzLabelPoints[below=3pt](A,B,L,M)
\tkzDrawPoints(M,N,K,L)
\end{tikzpicture}
\end{center}

\begin{hint}
Il y a plusieurs triangles dans lesquels on peut appliquer le théorème de Thalès.
\end{hint}
\begin{sol}
On a $MB=DN$ donc $MBND$ est un parallélogramme.

On en déduit que $(DM)//(BN)$.

Comme $M$ est le milieu de $[AB]$, on a par le théorème des milieux que $AK=KL$.

De la même façon, le théorème de Thalès appliqué dans $DCK$ entraîne que $KL=LC$, d'où le résultat.

\underline{Variante, demande plus d'initiative} :\\
Soit $Q$ le symétrique de $M$ par rapport à $B$. On a $MB=BQ$, donc $BQ=NC$ et $(BQ)//(NC)$. Donc $BQCM$ est un parallélogramme et donc $(NB)//(CQ)$.

Comme $AM=MB=BQ$ et que $(MK)//(BL)//(QC)$, le théorème de Thalès donne $AK=KL=LC$.

\underline{Autre preuve, avec centre de gravité} :\\
Dans le triangle $ABD$, le point $K$ est l'intersection des deux médianes $(DM)$ et $(AO)$. C'est donc le centre de gravité de $ABD$.

On en déduit que $KA = 2KO$.

Par symétrie centrale de centre $O$, on a $AK=CL$ et $KO=LO$, et finalement $AK = KO+OL = KL = LC$.
\end{sol}
\end{exo}


% - - - - - - - -
% - - - - - - - -
% - - - - - - - -
\begin{exo}[Un autre partage en trois]
% assez difficile
% symétrie centrale, homothéthie, Thalès, barycentres
Soit $ABCD$ un parallélogramme, $K$ le milieu de $[AD]$, $L$ le milieu de $[BC]$.
Les diagonales du parallélogramme $ABLK$ se coupent en $G$.
Montrer que les droites $(CG)$ et $(DG)$ coupent $[AB]$ deux points $I$ et $J$ qui le partagent en trois parties égales.
\begin{center}
\begin{tikzpicture}[rotate=10]
\tkzDefPoints{-3/1/A, -2/-1/B, 3/-1/C, 2/1/D}
\tkzDefMidPoint(A,D)\tkzGetPoint{K}
\tkzDefMidPoint(B,C)\tkzGetPoint{L}
\tkzInterLL(B,K)(A,L)\tkzGetPoint{G}
\tkzInterLL(C,G)(A,B)\tkzGetPoint{I}
\tkzInterLL(D,G)(A,B)\tkzGetPoint{J}

\tkzDrawPolygon(A,B,C,D)
\tkzDrawSegments(A,L K,B)
\tkzDrawSegments[dashed](D,J C,I)
\tkzDrawPoints(K,L,G,I,J)
\tkzMarkSegments[mark=||](A,K K,D B,L L,C)
\tkzLabelPoints[below=3pt](B,L,C)
\tkzLabelPoints[above=3pt](A,K,D,G)
\tkzLabelPoints[left=3pt](I,J)
\end{tikzpicture}
\end{center}
\begin{hint}
Tracer le symétrique de $K$ par rapport à $A$, et le symétrique de $L$ par rapport à $D$.
\end{hint}
\end{exo}




% - - - - - - - -
% - - - - - - - -
% - - - - - - - -
\begin{exo}[Médiane sur médiane]
% source : http://ressources.unisciel.fr/sillages/mathematiques/geometrie/res/Barycentres.pdf
% faisable avec barycentres

Soit $ABC$ un triangle, $M$ le milieu de $[BC]$ et $N$ le milieu de $[AM]$. À quel endroit la droite $(CN)$ coupe-t-elle le segment $[AB]$ ?
\begin{center}
\begin{tikzpicture}
\tkzDefPoints{1/4/A,-3/0/B, 3/0/C}
\tkzDefMidPoint(B,C)\tkzGetPoint{M}
\tkzDefMidPoint(A,M)\tkzGetPoint{N}
\tkzDrawPolygon(A,B,C)
\tkzDrawSegment(A,M)
\tkzMarkSegments[mark=||](B,M M,C)
\tkzMarkSegments[mark=|](M,N N,A)

\tkzLabelPoints[below=3pt](B,M,C)
\tkzLabelPoints[above=3pt](A)
\tkzLabelPoints[above right](N)
\tkzInterLL(C,N)(A,B)\tkzGetPoint{X}
\tkzDrawPoints(M,N)

\tkzDrawLine[dashed](C,X)
\tkzDrawPoint[fill=white](X)
\tkzLabelPoint[above=3pt](X){$?$}
\end{tikzpicture}
\end{center}

\begin{hint}
Tracer la parallèle à $(PC)$ passant par $M$.
\end{hint}
\begin{sol}
La parallèle à $(PC)$ passant par $M$ coupe le segment $[AB]$ en un point $Q$. D'après le théorème de Thalès dans le triangle $AQM$, on a : 
\[ \frac{AP}{AQ}=\frac{AN}{AM}.\]
On en déduit que  $P$ est le milieu du segment $[AQ]$.

Appliquons maintenant le théorème de Thalès dans le triangle $BPC$ : on obtient alors
\[\frac{BM}{BC}=\frac{BQ}{BP}, \]
et donc $Q$ est le milieu du segment $[BP]$.

On en déduit donc que $BQ=QP=PA$, et donc la droite $(CN)$ coupe le segment $[AB]$ aux deux tiers.
\end{sol}
\end{exo}




% - - - - - - - -
% - - - - - - - -
\begin{exo}[Échange médiane contre hauteur, v2 avec rotations]%vecten, moulin à vent
Sur l'extérieur d'un triangle $ABC$, on accole des carrés $ABIJ$ et $BCKL$.
On note $P$ le milieu du segment$[IL]$.
Montrer que la droite $(PB)$ est perpendiculaire à $(AC)$ et que de plus $AC=2PB$.
\begin{center}
\begin{tikzpicture}
\tkzDefPoint(0,0){A}
\tkzDefPoint(1,3){B}
\tkzDefPoint(4,0){C}
\tkzDefPoint(0,0){O}
\tkzDefSquare(A,B) \tkzGetFirstPoint{I}\tkzGetSecondPoint{J}
\tkzDefSquare(B,C) \tkzGetFirstPoint{K}\tkzGetSecondPoint{L}
%\tkzDrawPolygon(A,B,C)
\tkzDrawPolygon(A,B,I,J)
\tkzDrawPolygon(B,C,K,L)

\tkzDefMidPoint(I,L)\tkzGetPoint{P}
\tkzDrawSegment(I,P)
\tkzDrawSegment(P,L)
\tkzMarkSegment[mark=s||](P,L)
\tkzMarkSegment[mark=s||](I,P)
\tkzDrawSegment(A,C)
\tkzDrawSegment[dashed](P,B)
\tkzAutoLabelPoints[center=B](A,C,I,J,K,L,P)
\tkzLabelPoints[left](B)
\end{tikzpicture}
\end{center}
\begin{hint}
Considérer une rotation de $90^\circ$ de centre $B$, appliquée au triangle $ABC$.
\end{hint}
\begin{sol}
ATTENTIOn finir figure.


La rotation d'angle $90^\circ$ et de centre $B$ envoie le triangle $ABC$ sur le triangle $IBM$, où $M$ est le symétrique de $L$ par rapport à $B$.

On en déduit que $IM=AC$ et que $(IM)\bot(AC)$.
Maintenant, dans le triangle $ILM$, on applique le théorème des milieux avec le segment $[PB]$.

On aurait pu appliquer la rotation à $IBL$, ça marche aussi.
On peut aussi considérer la rotation de centre $B$ et d'angle $-90^\circ$, ça marche aussi.
\end{sol}
\end{exo}


% - - - - - - - -
% - - - - - - - -
% - - - - - - - -
\begin{exo}[Deux triangles isocèles rectangles REMPLACER par VECTEN] % exo7
% rotations
Soient $AOB$ et $COD$ deux triangles directs, isocèles rectangles en $O$. Soient $I$, $J$, $K$ et $L$ les milieux des segments $[AB]$, $[BC]$, $[CD]$ et $[DA]$.
\begin{center}
\definecolor{uuuuuu}{rgb}{0.26666666666666666,0.26666666666666666,0.26666666666666666}
\definecolor{qqqqff}{rgb}{0.,0.,1.}
\begin{tikzpicture}[line cap=round,line join=round,>=triangle 45,x=1.0cm,y=1.0cm]
\clip(-4.42,-2.62) rectangle (7.18,6.76);
\draw (-3.5,1.02)-- (0.48,1.46);
\draw (0.48,1.46)-- (0.04,5.44);
\draw (0.04,5.44)-- (-3.5,1.02);
\draw (2.2,0.)-- (0.48,1.46);
\draw (0.48,1.46)-- (1.94,3.18);
\draw (1.94,3.18)-- (2.2,0.);
\draw [dash pattern=on 2pt off 2pt] (-3.5,1.02)-- (1.94,3.18);
\draw [dash pattern=on 2pt off 2pt] (2.2,0.)-- (0.04,5.44);
\draw (-3.5,1.02)-- (2.2,0.);
\draw (1.94,3.18)-- (0.04,5.44);
\draw (-0.65,0.51)-- (-1.73,3.23);
\draw (-1.73,3.23)-- (0.99,4.31);
\draw (0.99,4.31)-- (2.07,1.59);
\draw (2.07,1.59)-- (-0.65,0.51);
\begin{scriptsize}
\draw [fill=qqqqff] (0.48,1.46) circle (2.5pt);
\draw[color=qqqqff] (0.2,1.15) node {$O$};
\draw [fill=qqqqff] (-3.5,1.02) circle (2.5pt);
\draw[color=qqqqff] (-3.68,1.51) node {$A$};
\draw [fill=qqqqff] (0.04,5.44) circle (2.5pt);
\draw[color=qqqqff] (0.18,5.81) node {$B$};
\draw [fill=qqqqff] (1.94,3.18) circle (2.5pt);
\draw[color=qqqqff] (2.08,3.55) node {$C$};
\draw [fill=qqqqff] (2.2,0.) circle (2.5pt);
\draw[color=qqqqff] (2.34,0.37) node {$D$};
\draw [fill=uuuuuu] (-1.73,3.23) circle (1.5pt);
\draw[color=uuuuuu] (-2.34,3.33) node {$I$};
\draw [fill=uuuuuu] (0.99,4.31) circle (1.5pt);
\draw[color=uuuuuu] (1.22,4.69) node {$J$};
\draw [fill=uuuuuu] (2.07,1.59) circle (1.5pt);
\draw[color=uuuuuu] (2.3,1.79) node {$K$};
\draw [fill=uuuuuu] (-0.65,0.51) circle (1.5pt);
\draw[color=uuuuuu] (-0.8,0.03) node {$L$};
\end{scriptsize}
\end{tikzpicture}
\end{center}

%Montrer que $(AC)\bot (BD)$ et que $AC=BD$.
Montrer que $IJKL$ est un carré. 
\begin{hint}
Penser au problème \ref{pseudo-carre}.
\end{hint}
\begin{sol}
ATTENTION VIRER LES VECTEURS.
On commence par prouver que $(AC)\bot (BD)$ et que $AC=BD$. Ensuite on rouve le résultat.

\begin{enumerate}
\item Soit $\rho$ la rotation de centre $O$ et d'angle $\pi/2$. D'après l'énoncé, on a $\rho(B)=A$ et $\rho(D)=C$. Donc $[AC]$ est l'image de $[AD]$ par $\rho$, d'où on déduit que $(AC)\bot (BD)$ et que $AC=BD$.

\item Le quadrilatère $IJKL$ est toujours un parallélogramme, même sans hypothèses sur $ABCD$ (c'est le théorème de Varignon). En effet, par le théorème de Thalès (ou simplement le théorème des milieux) : 
\[ \overrightarrow{IL} = \frac12 \overrightarrow{BD} = \overrightarrow{JK}.\]
Ceci signifie que les côtés $[IL]$ et $[JK]$ ont même longueur et sont parallèles, donc $IJKL$ est un parallélogramme.

Pour voir que c'est un carré, remarquons qu'on montre de même que  
\[ \overrightarrow{IJ} = \frac12 \overrightarrow{AC} = \overrightarrow{LK},\]
et d'après la première question, $(AC)\bot (BD)$ et que $AC=BD$, donc $IJKL$ est un parallélogramme ayant un angle droit et des cotés consécutifs de même longueur. C'est donc un carré.
\end{enumerate}
\end{sol}
\end{exo}


\begin{exo}[Le triangle de Feynman/partage en sept]

Soit $ABC$ un triangle, 


(séparer les deux étapes, voir page wikipedia "partage d'un triangle en sept" .

Autres exos avec tritianes ?
\end{exo}



% - - - - - - - - - -
\begin{exo}
Énoncé
\begin{hint}
Indication.
\end{hint}
\begin{sol}
Correction.
\end{sol}
\end{exo}

% - - - - - - - - - -
\begin{exo}
Énoncé
\begin{hint}
Indication.
\end{hint}
\begin{sol}
Correction.
\end{sol}
\end{exo}


Petit exo marrant:

\url{https://www.youtube.com/watch?v=wIk97NRwj5I}


\section{Thalès, version générale mais non croisé}

TODO : rajouter \url{https://www.youtube.com/watch?v=l6XaE_U9FDU} Thalès, mais aussi triangles semblables. Déplacer dans triangles semblables ?




% - - - - - - - - - -
\begin{exo}
%[LH271 (2nde)]
Soit $ABC$ un triangle.
Pour tout point $D$ de $[BC]$, on construit les points $E$ et $F$ comme suit : $E$ est le point où la parallèle à $(AB)$ passant par $D$ recoupe $(AC)$, et $F$ est le point où la parallèle à $(AC)$ passant par $D$ recoupe $(AB)$.
Comment faut-il choisir $D$ pour que $(EF)$ et $(BC)$ soient parallèles ?
\begin{hint}
\end{hint}
\begin{sol}
On applique deux fois Thalès, on voit que c'est parallèle ssi $D$ est le milieu  de $[BC]$.
\end{sol}
\end{exo}



% - - - - - - - - - -
\begin{exo}
%LH 272
Soit $ABC$ un triangle et $G$ son centre de gravité.
Les parallèles à $(AB)$ et $(AC)$ passant par $G$ recoupent $[BC]$ en $D$ et $E$.
Que peut-on dire des segments $[BD]$, $[DE]$ et $[EC]$ ?
\begin{hint}
\end{hint}
\begin{sol}
Les trois segments sont égaux. Mettre dans la section théorème des milieux ?
\end{sol}
\end{exo}

% - - - - - - - - - -
\begin{exo}
LH 278 basique deux fois Thalès.
\begin{hint}
Indication.
\end{hint}
\begin{sol}
Correction.
\end{sol}
\end{exo}

% - - - - - - - - - -
\begin{exo}[Double hauteur]
Soit $ABC$ un triangle.
On note $D$ et $E$ les pieds des hauteurs issues de $B$ et de $C$.
On note ensuite $F$ et $G$ les pieds des hauteurs du triangle $ADE$ issues de $D$ et de $E$.
Montrer que $(FG)//(BC)$.
\begin{hint}
Deux fois Thalès direct et on conclut avec la réciproque.
\end{hint}
\begin{sol}
% Source : LH 276, mignon
\end{sol}
\end{exo}


% - - - - - - - - - -
\begin{exo}[Tourniquet, version 1]
Soit $ABC$ un triangle et $M$, $M'$ deux points quelconques sur $[BC]$.
Par $M$ et $M'$ on mène les parallèles à $(AB)$, qui recoupent $(AC)$ en $P$ et $P'$.
Toujours par $M$ et $M'$, on mène les parallèles à $(AC)$, qui recoupent $(AB)$ en $Q$ et $Q'$.
Montrer que $(QP)//(Q'P')//(BC)$.
\begin{hint}
Indication.
\end{hint}
\begin{sol}
% Voir aussi LH 281 mais pas tt à fait pareil.
\end{sol}
\end{exo}

% - - - - - - - - - -
\begin{exo}
Énoncé
\begin{hint}
Indication.
\end{hint}
\begin{sol}
Correction.
\end{sol}
\end{exo}

% - - - - - - - - - -
\begin{exo}
Énoncé
\begin{hint}
Indication.
\end{hint}
\begin{sol}
Correction.
\end{sol}
\end{exo}



% - - - - - - - - - -
\begin{exo}
Énoncé
\begin{hint}
Indication.
\end{hint}
\begin{sol}
Correction.
\end{sol}
\end{exo}



% - - - - - - - -
% - - - - - - - -
% - - - - - - - -
\begin{exo}[Thalès, variation 1]
Soit $ABC$ un triangle, $M$ un point de $[AB]$ et $N$ un point de $[AC]$ tels que $(MN)//(BC)$. Montrer que 
\[ \frac{AM}{MB} = \frac{AN}{NC}.\]
\end{exo}


% - - - - - - - -
%% - - - - - - - -
% - - - - - - - -
%\begin{exo}[Subdivision en sept \icoRegle\icoCompas]
%[Subdivision]
%On donne un segment $[AB]$. Le diviser en sept parts égales.
%\end{exo} 





% - - - - - - - -
% - - - - - - - -
% - - - - - - - -
\begin{exo}[Thalès, variation 2]
Soient $A$, $B$, $C$ trois points distincts alignés sur une droite $\mathcal D$, et $A'$, $B'$, $C'$ trois autres points distincts alignés sur une autre droite $\mathcal D'$, tels que l'on ait le parallélisme suivant
\[ (AA')// (BB')//(CC').\]
\begin{center}
\begin{tikzpicture}[rotate=-10,xscale=.6,yscale=.4]
\tkzDefPoints{0/0/A, -2/-4/B, -3/-6/C, 3/0/A', 7/-4/B', 9/-6/C'}
\tkzDrawLines(A,A' B,B' C,C' A,C A',C')
\tkzLabelPoints[above left](A,B,C)
\tkzLabelPoints[above right](A',B',C')
\end{tikzpicture}
\end{center}
Montrer que l'on a la relation suivante, qui est une forme généralisée du théorème de Thalès:
\[ \frac{AB}{AC} = \frac{A'B'}{A'C'}\]
Cette version est souvent très utile.
\begin{hint}
Tracer une nouvelle droite pour pouvoir appliquer le théorème de Thalès habituel.
\end{hint}
\begin{sol}
Traçons la parallèle à $\mathcal D'$ passant par $A$. Elle coupe $(BB')$ en $B''$ et $(CC')$ en $C''$. 

\begin{center}
\begin{tikzpicture}
\end{tikzpicture}
\end{center}

Le théorème de Thalès donne alors la relation
\[ \frac{AB}{AC} = \frac{AB''}{AC''}.\]
D'autre part, le parallélisme entraîne que 
\[ AB'' = A'B' \quad \text{ et }\quad AC''=A'C'.\]
On en déduit que 
\[ \frac{AB}{AC} = \frac{A'B'}{A'C'}.\]

\end{sol}
\end{exo}

% - - - - ---  - 
% - - - - - - - -
% - - - - - - - -
\begin{exo}[Théorème de la bissectrice]

Soit $ABC$ un triangle, et $M$ le pied de la bissectrice issue de $A$. Montrer que $\frac{AB}{AC}=\frac{MB}{MC}$.
\begin{center}
\begin{tikzpicture}[scale=.8]
\tkzDefPoints{1/4/A, 0/0/B, 6/0/C}
\tkzDrawPolygon[very thick](A,B,C)
\tkzDefLine[bisector](B,A,C) \tkzGetPoint{x}
\tkzInterLL(A,x)(B,C)\tkzGetPoint{M}
\tkzDrawSegment[dashed](A,M)
\tkzDrawPoint[fill=white](M)
\tkzMarkAngles[mark=|](B,A,M M,A,C)
\tkzLabelPoints[above](A)
\tkzLabelPoints[below](B,M,C)
\end{tikzpicture}
\end{center}
\begin{hint}
Tracer la parallèle à la bissectrice passant par un des sommets.
\end{hint}
\begin{sol}
\end{sol}
\end{exo}

\begin{exo}[TODO Applications du théorème de la bissectrice]
% https://artofproblemsolving.com/wiki/index.php/Angle_bisector_theorem trois exos
\end{exo}


% - - - - - - - -
% - - - - - - - -
% - - - - - - - -
\begin{exo}[FIG Pappus affine]
% Source :  Audin par exemple 
% homothéties
Soient $D$ et $D'$ deux droites non parallèles.
Soient $A, B, C$ trois points sur $D$, et $A'$, $B'$ et $C'$ trois points sur $D'$.
On suppose que $(AB') // (BC')$ et $(BA') // (CB')$.
Montrer que $(AA') // (CC')$.
\begin{hint}
On peut appliquer le théorème de Thalès dans deux triangles différents.
\end{hint}
\begin{sol}
Pour démontrer le parallélisme, d'après la réciproque du théorème de Thalès, il suffit de démontrer l'égalité suivante:
\[\frac{OC}{OA}=\frac{OC'}{OA'}.\]

Appliquons le théorème de Thalès dans le triangle $OBA'$. On obtient
\[ \frac{OC}{OB} = \frac{OB'}{OA'}.\]
Appliquons ensuite le théorème de Thalès dans le triangle $OAB'$. On obtient cette fois
\[ \frac{OB}{OA} = \frac{OC'}{OB'}.\]

On peut alors écrire:
\[ \frac{OC}{OA}
=\frac{OC}{OB}\frac{OB}{OA} 
= \frac{OB'}{OA'}\frac{OC'}{OB'}
=  \frac{OC'}{OA'},\]
ce qu'il suffisait de démontrer.
\end{sol}
\end{exo}  





% - - - - - - - - - -
\begin{exo}[Desargues affine]
Soit $G$ le centre de gravité d'un triangle $ABC$.
On mène par le point $G$ les parallèles à $(AB)$ et $(AC)$, qui coupent la droite $(BC)$ en $D$ et $E$.
Comparer les trois segments $[BD]$, $[DE]$ et $[EC]$.
\begin{hint}
Où se trouve le centre de gravité sur les médianes ?
\end{hint}
\begin{sol}
Les trois segments sont égaux.
% source : LH3ème exo 14
\end{sol}
\end{exo}


% - - - - - - - - - -
\begin{exo}
Soit $ABCD$ un quadrilatère et $O$ le point d'intersection de ses diagonales.
Les parallèles à $(BC)$ et $(CD)$ passant par $O$ coupent les droites $(AD)$ et $(AD)$ en $E$ et $F$.
Montrer que $(EF)//(BD)$.
\begin{hint}
Indication.
\end{hint}
\begin{sol}
% LH3ème ex15
\end{sol}
\end{exo}


% - - - - - - - - - -
\begin{exo}
Soit $ABCD$ un parallélogramme.
On mène par le point $A$ une sécante qui recoupe les droites $(BD)$, $(BC)$ et $(CD)$ en $M$, $P$ et $Q$.
Montrer que $MA^2=MP\times MQ$.
\begin{hint}
Indication.
\end{hint}
\begin{sol}
% LH3ème ex17
\end{sol}
\end{exo}



% - - - - - - - - - -
\begin{exo}
Soit $ABC$ un triangle avec $\frac{AB}{AC} = \frac{3}{2}$.
Expliquer comment placer un point $M$ entre $B$ et $C$ de sorte que $\frac{BM}{MC}=\frac{3}{2}$.
La parallèle à $(AM)$ passant par $C$ recoupe $(AB)$ en $D$.
Comparer les angles $\widehat{BAM}$ et $\widehat{MAC}$.
\begin{hint}
Indication.
\end{hint}
\begin{sol}
%LH3ème-ex19
\end{sol}
\end{exo}



% - - - - - - - - - -
\begin{exo}
Soient $[AB]$ et $[AC]$ deux cordes d'un cercle et $[IJ]$ le diamètre perpendiculaire à $[BC]$.
La droite $(IJ)$ recoupe $(AB)$ en $E$ et $(AC)$ en $F$.
Que représentent $(AI)$ et $(AJ)$ pour l'angle $\widehat{BAC}$ ?
Comparer les rapports $\dfrac{IE}{IF}$ et $\dfrac{JE}{JF}$. 
\begin{hint}
Indication.
\end{hint}
\begin{sol}
% LH 3ème 23 p.143
\end{sol}
\end{exo}




% - - - - - - - - - -
\begin{exo}[Triangle orthique]
Soit $ABC$ un triangle.
On note $A'$, $B'$ et $C'$ les pieds des hauteurs.
Montrer que $(AA')$ est la bissectrice de l'angle $\widehat{B'A'C'}$. % avec ou sans angle inscrit ?


Notons ensuite  $H$ l'orthocentre du triangle et $P$ l'intersection de $(AA')$ et $(B'C')$. 
Comparer les rapports $\dfrac{AP}{AA'}$ et $\dfrac{HP}{HA'}$.
\begin{hint}
Indication.
\end{hint}
\begin{sol}
% LH 24 avec le théorème de la bissectrice et sans angle inscrit !
\end{sol}
\end{exo}



% - - - - - - - - - -
\begin{exo}
Soit $\mathcal C$ un cercle de centre $O$ et $B$, $C$ deux points non diamétralement opposés de ce cercle.
Les tangentes en $B$ et $C$ se recoupent en $A$ et la droite $(OA)$ recoupe le cercle en $D$ puis en  $E$ (de sorte que $D$ est entre $A$ et $E$).
Enfin, on note $H$ l'intersection de $(AO)$ et de la corde $[BC]$.
Que représentent $(BD)$ et $(BE)$ pour l'angle $\widehat{ABC}$ ?% attention il faut l'angle inscrit ??
Comparer $\dfrac{AE}{AD}$ et $\dfrac{HE}{HD}$.
\begin{center}
\begin{tikzpicture}[scale=3]
\tkzDefPoint(0,0){O}\tkzDefPoint(-50:1){B} \tkzDefPoint(85:1){C}
\tkzDefPointBy[rotation=center B angle 90](O) \tkzGetPoint{B'}
\tkzDefPointBy[rotation=center C angle 90](O) \tkzGetPoint{C'}
\tkzInterLL(B,B')(C,C')\tkzGetPoint{A}
\tkzInterLL(B,C)(O,A)\tkzGetPoint{H}
\tkzInterLC(O,A)(O,B)\tkzGetPoints{E}{D}

\tkzDrawLines[very thick](A,B A,C)
\tkzDrawLine(A,E)
\tkzDrawCircle[very thick](O,B)
\tkzDrawSegment(B,C)
\tkzDrawSegments[dashed](B,D B,E)

\tkzDrawPoints(O,B,C,A,D,E,H)
\tkzLabelPoints(O,A,B,C,D,E,H)
\end{tikzpicture}
\end{center}
\begin{hint}
Indication.
\end{hint}
\begin{sol}
Correction.
\end{sol}
\end{exo}



% - - - - - - - - - -
\begin{exo}
Énoncé
\begin{hint}
Indication.
\end{hint}
\begin{sol}
Correction.
\end{sol}
\end{exo}







% - - - - - - - -
% - - - - - - - -
% - - - - - - - -
\begin{exo}[Construction d'un carré inscrit dans un triangle]
% homothétie, Thalès
Soit $ABC$ un triangle.
%\begin{multicols}{2}

On souhaite construire un carré intérieur à $ABC$ dont un sommet appartienne à $[AB]$, un à $[AC]$ et deux sommets adjacents appartiennent à $(BC)$. On procède en trois étapes:

\noindent\textbf{Étape 1} On construit un grand carré extérieur $BCDE$.\\
\textbf{Étape 2}  On  trace les droites $(AE)$ et $(AD)$. Elles  coupent le segment $[BC]$ en deux points $I$ et $J$.\\
\textbf{Étape 3}  On trace les perpendiculaires à $(BC)$ passant par $I$ et $J$. Elles coupent $[AC]$ et $[AB]$ en $K$ et $L$.\\

Montrer que le quadrilatère $IJKL$ ainsi construit est effectivement un carré.
\begin{center}
\begin{tikzpicture}
\tkzDefPoints{1.5/4/A, 0/0/B, 4/0/C, 4/-4/D, 0/-4/E}
\tkzDrawPolygon[very thick](A,B,C)
\tkzDrawPolygon(B,C,D,E)
\tkzInterLL(A,E)(B,C)\tkzGetPoint{I}
\tkzInterLL(A,D)(B,C)\tkzGetPoint{J}
\tkzDefSquare(I,J)\tkzGetFirstPoint{K}\tkzGetSecondPoint{L}

\tkzDrawSegments[dashed](A,E A,D J,K I,L)
\tkzLabelPoints[above](A)
\tkzLabelPoints[left](B,E,L)
\tkzLabelPoints[right](C,D,K)
\tkzLabelPoints[below right](I)
\tkzLabelPoints[below left](J)
\tkzDrawPoints(A,B,C,D,E,I,J,K,L)
\tkzMarkRightAngles[size=.3](L,I,B K,J,C)
\end{tikzpicture}
\end{center}
%\end{multicols}
\begin{hint}   

\end{hint}      
\begin{sol} 

Analyse. Traçons comme suggéré une figure avec le carré déjà construit : on trace un carré puis on trace un triangle adéquat autour. On constate qu'un des côtés du carré, notons-le $[IJ]$, est parallèle à $[BC]$. Il y a une homothétie $h$ de centre $A$ qui envoie $[IJ]$ sur $[BC]$. Alors, l'image du carré $IJKL$ par $h$ est un carré dont un des côtés est $[BC]$. Notons $BCDE$ ce carré et traçons-le. On constate que $h(K)=D$ et $h(L)=E$, c'est-à-dire $K=h^{-1}(D)$ et $L = h^{-1}(E)$. Il ne reste plus qu'à faire la synthèse.


\end{sol}  
\end{exo}  



% - - - - - - - -
% - - - - - - - -
\begin{exo}[Théorème de Ceva]
Soit $ABC$ un triangle quelconque. Soient $I$, $J$ et $K$ trois points placés sur chaque côté de ce triangle, comme sur la figure.
Montrer que les droites $(AI)$, $(BJ)$ et $(CK)$ sont concourantes si et seulement si on a l'égalité
\[ \frac{IB}{IC} \times \frac{JC}{JA} \times \frac{KA}{KB} = 1.\]
En déduire une nouvelle preuve du fait que les médianes d'un triangle sont concourantes :-)
\begin{hint}
\end{hint}
\begin{sol}
\end{sol}
\end{exo}


% - - - - - - - -
% - - - - - - - -
\begin{exo}[Application de Ceva : le point de Gergonne]
Soit $ABC$ un triangle et $I$, $J$ $K$ les points de contact avec le cercle inscrit.
Montrer que les droites $(AI)$, $(BJ)$ et $(CK)$ sont concourantes.
\begin{center}
\definecolor{uuuuuu}{rgb}{0.26666666666666666,0.26666666666666666,0.26666666666666666}
\definecolor{ududff}{rgb}{0.30196078431372547,0.30196078431372547,1.}
\begin{tikzpicture}[line cap=round,line join=round,>=triangle 45,x=1.0cm,y=1.0cm,rotate=70]

\draw [very thick] (-1.,0.)-- (1.,3.) --  (7.,0.) -- cycle;
\draw  (1.44867367148231,1.3104903788145046) circle (1.3104903788145046cm);
\draw [dashed] (1.,3.)-- (1.44867367148231,0.);
\draw [dashed] (2.0347427856600464,2.4826286071699766)-- (-1.,0.);
\draw [dashed] (0.3582797660627899,2.037419649094185)-- (7.,0.);
% rayons
%\draw [dashed] (1.44867367148231,1.3104903788145046)-- (2.0347427856600464,2.4826286071699766);
%\draw [dashed](1.44867367148231,1.3104903788145046)-- (0.3582797660627899,2.037419649094185);
%\draw [dashed] (1.44867367148231,1.3104903788145046)-- (1.44867367148231,0.);

\draw(-1.,0.) node {$\bullet$};
\draw (-1.58,-0.11) node {$A$};
\draw  (7.,0.) node {$\bullet$};
\draw (7.14,0.37) node {$B$};
\draw  (1.,3.) node {$\bullet$};
\draw (1.14,3.37) node {$C$};
\draw  (1.44867367148231,0.) node {$\bullet$};
\draw (1.6,-0.43) node {$C'$};
\draw  (2.0347427856600464,2.4826286071699766) node {$\bullet$};
\draw (2.24,2.81) node {$A'$};
\draw  (0.3582797660627899,2.037419649094185) node {$\bullet$};
\draw (-0.02,2.41) node {$B'$};
%\draw [fill=uuuuuu] (1.44867367148231,1.3104903788145046) circle (2.0pt);
%\draw[color=uuuuuu] (1.6,1.09) node {$I$};
\draw (1.1817406392361596,1.7848141035171592) node[fill=white] {$??$};
%\draw[color=uuuuuu] (1.32,2.11) node {$E$};
\end{tikzpicture}
\end{center}
Indication : utiliser le théorème de Ceva.
\begin{hint}
\end{hint}
\begin{sol}
\end{sol}
\end{exo}



% - - - - - - - -
% - - - - - - - -
\begin{exo}[Application de Ceva : le point de Nagel]
On considère un triangle $ABC$ ainsi que ses trois cercles exinscrits (voir problème précédent). O note $A'$, $B'$ et $C'$ les points de contact de ces cercles avec les côtés du triangle, comme sur la figure. Montrer que les trois droites $(AA')$, $(BB')$ et $(CC')$ sont concourantes. 
\begin{center}
\definecolor{ududff}{rgb}{0.30196078431372547,0.30196078431372547,1.}
\definecolor{xdxdff}{rgb}{0.49019607843137253,0.49019607843137253,1.}
\definecolor{uuuuuu}{rgb}{0.26666666666666666,0.26666666666666666,0.26666666666666666}
\begin{tikzpicture}[line cap=round,line join=round,>=triangle 45,x=1.0cm,y=1.0cm]
\clip(-1.84,-2.7) rectangle (4.94,3.4);
\draw  (1.7961795736231998,-2.906279600020632) circle (2.906279600020632cm);
\draw (4.03224755112299,2.4920660376475365) circle (2.4920660376475365cm);
\draw  (-1.0322475511229898,1.6702116225208423) circle (1.6702116225208423cm);
\draw [dashed] (1.,2.)-- (1.7961795736232002,0.);
\draw [dashed] (0.,0.)-- (2.2700907567377264,0.7299092432622734);
\draw [dashed] (3.,0.)-- (0.46163513878373896,0.9232702775674779);
\draw  (-2.24,0.)-- (5.76,0.);
\draw  (5.19,-2.19)-- (-0.67,3.67);
\draw  (2.476,4.952)-- (-1.464,-2.928);

\draw  (0.,0.) node {$\bullet$};
\draw (-0.54,-0.19) node {$A$};
\draw  (3.,0.)node {$\bullet$};
\draw (3.54,-0.19) node {$B$};
\draw (1.,2.) node {$\bullet$};
\draw (0.8,2.65) node {$C$};
\draw  (1.7961795736232002,0.) node {$\bullet$};
\draw (1.82,-0.33) node {$C'$};
\draw  (2.2700907567377264,0.7299092432622734) node {$\bullet$};
\draw(2.48,1.05) node {$A'$};
\draw (0.46163513878373896,0.9232702775674779) node {$\bullet$};
\draw (0.14,1.07) node {$B'$};
\draw  (1.5923591472464007,0.5119961202954946) node[fill=white] {$??$};
%\draw[color=uuuuuu] (1.74,0.85) node {$N$};

\end{tikzpicture}
\end{center}
\begin{hint}
\end{hint}
\begin{sol}
\end{sol}
\end{exo}


% - - - - - - - -
% - - - - - - - -
\begin{exo}[Le miraculeux cercle des neuf points]
\def\neufpoints{
\tkzDefPoints{0/0/A,6/0/B,1/4/C}

\tkzDefTriangleCenter[centroid](A,B,C)
\tkzGetPoint{G}
\tkzDefSpcTriangle[medial](A,B,C){I,J,K}

\tkzDefSpcTriangle[orthic](A,B,C){H_A,H_B,H_C}
\tkzDefTriangleCenter[ortho](B,C,A)
\tkzGetPoint{H}

\tkzDrawSegments[thin](A,H_A B,H_B C,H_C)
\tkzMarkRightAngles[fill=gray!20,
opacity=.5](A,H_A,C B,H_B,A C,H_C,A)

\tkzDefMidPoint(A,H)\tkzGetPoint{X}
\tkzDefMidPoint(B,H)\tkzGetPoint{Y}
\tkzDefMidPoint(C,H)\tkzGetPoint{Z}
\tkzMarkSegments[size=1.5pt,mark=|](A,X X,H)
\tkzMarkSegments[size=1.5pt,mark=|||](B,Y Y,H)
\tkzMarkSegments[size=1.5pt,mark=||](C,Z Z,H)
\tkzDrawPolygon(A,B,C)
\tkzDrawPoints[size=5pt](I,J,K,H_A,H_B,H_C,X,Y,Z)
\tkzDrawPoints[fill=white,draw=black](H)

}
Soit $ABC$ un triangle.
On place:
\begin{itemize}
\item les milieux des trois côtés $I$, $J$ et $K$;
\item les pieds des hauteurs $H_A$ $H_B$ et $H_C$;
\item les points $M_A$, $M_B$ et $M_C$ situés à mi-chemin entre l'orthocentre $H$ et les sommets du triangle.
\end{itemize}
L'objectif de ce problème est de montrer la propriété époustouflante suivante : 
\begin{mdframed}
\noindent \textbf{Théorème} : les neuf points $I$, $J$, $K$, $H_A$, $H_B$, $H_C$, $M_A$, $M_B$ et $M_C$ sont sur un seul et même cercle.
\end{mdframed}
Ce cercle est appelé cercle des neuf points, cercle d'Euler, cercle de Feuerbach ou encore cercle de Terquem.
\begin{center}
\begin{tikzpicture}[scale=1.3]
\neufpoints{}
%\tkzAutoLabelPoints[center=G](A,B,C,I,J,K)
%\tkzAutoLabelPoints[center=H](A,B,C,I,J,K,H_A,H_B,H_C,X,Y,Z)
%\tkzDrawPolygon[dashed](I,J,X,Y)% rectangle
\end{tikzpicture}
\end{center}
Ce magnifique théorème se démontre en plusieurs étapes.

\paragraph{Étape 1}
Montrer que le quadrilatère $IJXY$ est un parallélogramme en appliquant le théorème de Thalès dans deux triangles différents.
\begin{center}
\begin{tikzpicture}[scale=1.7]
\neufpoints{}
\tkzAutoLabelPoints[center=G](A,B,C,I,J,K)
\tkzAutoLabelPoints[center=H](H_A,H_C)
\tkzLabelPoints[left=3pt](H_B)
\tkzDrawPolygon[dashed](I,J,X,Y)% rectangle
\tkzLabelPoints[below](X,Y)
\tkzLabelPoints[right](Z)
\end{tikzpicture}
\end{center}
\paragraph{Étape 1bis}
Montrer que le quadrilatère $IJXY$ est en réalité un rectangle et en déduire que $I$, $J$, $X$ et $Y$ se trouvent sur un même cercle $\mathcal C$.
\paragraph{Étape 2}
Montrer que les pieds des hauteurs $H_A$, et $H_B$ se trouvent également sur ce cercle $\mathcal C$. À ce stade il ne manque plus que trois points.
\paragraph{Étape 3}
En appliquant les étapes 1 et 1bis à un autre quadrilatère, en déduire que $Z$ et $K$ sont eux aussi sur le cercle $\mathcal C$.
\paragraph{Étape 4 et fin}
Comme dans l'étape 2, montrer que $H_C$ est sur le cercle.
\end{exo}



% - - - - - - - -
% - - - - - - - -
% - - - - - - - -
\begin{exo}[Desargues affine REFORMULER AVEC THALES]
% tags : homothéties.
\begin{enumerate}
\item Soient $ABC$ et $A'B'C'$ deux triangles (non aplatis) sans sommet commun. Montrer qu'ils se déduisent l'un de l'autre par homothétie ou translation ssi leurs côtés sont parallèles.
\item (Application) On donne deux droites se coupant en un point $O$ hors de la feuille, ainsi qu'un point $M$ hors de ces droites. Tracer la droite $(OM)$.
\end{enumerate}
\begin{hint}   
Homothéties et translations
\end{hint}
\end{exo}  








% - - - - - - - - - -
\begin{exo}
Énoncé
\begin{hint}
Indication.
\end{hint}
\begin{sol}
Correction.
\end{sol}
\end{exo}

% - - - - - - - - - -
\begin{exo}
Énoncé
\begin{hint}
Indication.
\end{hint}
\begin{sol}
Correction.
\end{sol}
\end{exo}

\section{Thalès, version générale avec croisements}

% - - - - - - - - - -
\begin{exo}
Soit $ABCD$ un trapèze.
Soient $M$ et $N$ les milieux des diagonales $[AC]$ et $[BD]$.
Montrer que la droite $(MN)$ est parallèle aux bases du trapèze.
\begin{hint}
Indication.
\end{hint}
\begin{sol}
Correction.
\end{sol}
\end{exo}

% - - - - - - - - - -
\begin{exo}[Diagonales d'un trapèze]

Soit $ABCD$ un trapèze de bases $(AD)$ et $(BC)$.
Soient $M$ et $N$ les milieux de ses diagonales $[AC]$ et $[BD]$.
Montrer que la droite $(MN)$ est parallèle aux bases, et qu'elle recoupe $[AB]$ et $[CD]$ en leur milieu.
\begin{hint}
Indication.
\end{hint}
\begin{sol}
Correction.
\end{sol}
\end{exo}

% - - - - - - - - - -
\begin{exo}[Bimédianes]
Soit $ABCD$ un quadrilatère convexe et soient $M$ et $N$ les milieux de ses deux diagonales.
Les \emph{bimédianes} de $ABCD$ sont les deux droites reliant les milieux de côtés opposés.
Montrer que les deux bimédianes et la droite $(MN)$ sont concourantes.
\begin{hint}
Indication.
\end{hint}
\begin{sol}
Source : \url{https://www.cut-the-knot.org/Curriculum/Geometry/AnyQuadri.shtml}
\end{sol}
\end{exo}

% - - - - - - - - - -
\begin{exo}[Olympiades de Moscou 1957]
Soit $ABCD$ un quadrilatère convexe, $M$ le milieu de $[AC]$ et $N$ le milieu de $[BD]$.
La droite $(MN)$ recoupe $(AB)$ en $M'$ et $(CD)$ en $N'$
Montrer que si $MM' = NN'$ alors $(BC)//(DA)$.
\begin{hint}
Indication.
\end{hint}
\begin{sol}
Source : \url{https://www.cut-the-knot.org/Curriculum/Geometry/GeoGebra/MidlineInQuadrilateral.shtml}
\end{sol}
\end{exo}

% - - - - - - - - - -
\begin{exo}
On the sides of ABC, three similar triangles, AKB, BLC, and CNA, are drawn outward. If AB and KL are bisected by D and E, respectively, prove that DE is parallel to NC and determine DE/NC.
\begin{hint}
Indication.
\end{hint}
\begin{sol}
Source : \url{https://www.cut-the-knot.org/Curriculum/Geometry/MidlineInSimTris.shtml}
\end{sol}
\end{exo}

\section{Théorème de Pappus affine, Desargues affine, et applications}

\begin{exo}[Pappus affine]
On place trois points $A$, $B$ et $C$ sur une droite $\mathcal D$ et trois autres points $A'$, $B'$ et $C'$ sur une autre droite $\mathcal D'$.
Montrer que si $(AB')//(BC')$ et $(BA')//(CB')$, alors $(AA')//(CC')$.
\begin{center}
\begin{tikzpicture}
\tkzDefPoint(0,0){A}\tkzDefPoint(3,0){B}\tkzDefPoint(4,0){C}
\tkzDefPoint(2,3){B'}\tkzDefPoint(6,4){X}

\tkzDefLine[parallel=through B](A,B') \tkzGetPoint{C''}
\tkzInterLL(B,C'')(B',X)\tkzGetPoint{C'}

\tkzDefLine[parallel=through B](C,B') \tkzGetPoint{A''}
\tkzInterLL(B,A'')(B',X)\tkzGetPoint{A'}

\tkzDrawLines(A,C A',C')
\tkzDrawLines(A,B' B,C' A',B B',C)
\tkzDrawLines[dashed](A,A' C,C')
\tkzDrawPoints(A,B,C,A',B',C')
\tkzLabelPoints[above](A,B,C,A',B',C')
\end{tikzpicture}
\end{center}
\begin{hint}
Distinguer selon si $\mathcal D$ et $\mathcal D'$ sont sécantes ou pas.
\end{hint}
\begin{sol}
\end{sol}
\end{exo}


\begin{exo}[Desargues affine]
Soient $ABC$ et $A'B'C'$ deux triangles sans côtés communs et dont les côtés sont deux à deux parallèles, autrement dit $(AB)//(A'B')$, $(BC)//(B'C')$ et $(CA)//(C'A')$.
Montrer que les trois droites $(AA')$, $(BB')$ et $(CC')$ sont concourantes ou parallèles.
\begin{center}
\begin{tikzpicture}
\end{tikzpicture}
\end{center}
\begin{hint}
Distinguer suivant si  $(AA')$ et $(BB')$ sont parallèles ou bien sécantes.
\end{hint}
\begin{sol}
\end{sol}
\end{exo}

\begin{exo}
\begin{center}
\begin{tikzpicture}
\end{tikzpicture}
\end{center}
\begin{hint}
\end{hint}
\begin{sol}
\end{sol}
\end{exo}

\begin{exo}
\begin{center}
\begin{tikzpicture}
\end{tikzpicture}
\end{center}
\begin{hint}
\end{hint}
\begin{sol}
\end{sol}
\end{exo}


\section{Théorème de Ceva et applications}

Recherche "By Ceva's theorem" .

Orthocentre mais il faut le cosinus. \url{https://math.stackexchange.com/questions/1472793/prove-the-lines-of-the-orthocenter-are-concurrent-by-cevas-theorem}

Centre du cercle inscrit mais il faut le sinus : \url{https://sms.math.nus.edu.sg/smsmedley/Vol-23-1/On%20Ceva%27s%20Theorem%20(Hang%20Kim%20Hoo%20and%20Koh%20KM).pdf} qui contient d'autres exos.

Théorème du papillon avec Ceva \url{https://forumgeom.fau.edu/FG2016volume16/FG201623.pdf}

Feuille du club math de UCLA \url{https://circles.math.ucla.edu/circles/lib/data/Handout-2746-2365.pdf}


Un exo de RMO \url{https://www.cheenta.com/rmo-2002-problem-1-cevas-theorem/}

Quelques exos sur \url{https://artofproblemsolving.com/wiki/index.php/Ceva%27s_theorem}

Très bien : \url{https://www.cut-the-knot.org/Generalization/ceva.shtml}

Point de Lemoine




\section{Théorème de Menelaüs et applications}



\begin{exo}[Menelaus]
\begin{center}
\begin{tikzpicture}
\end{tikzpicture}
\end{center}
\begin{hint}
\end{hint}
\begin{sol}
\end{sol}
\end{exo}

\begin{exo}[Pappus par Menelaus]
\begin{center}
\begin{tikzpicture}
\end{tikzpicture}
\end{center}
\begin{hint}
\end{hint}
\begin{sol}
% https://fr-academic.com/dic.nsf/frwiki/1633467
\end{sol}
\end{exo}

\begin{exo}[Desargues par Menelaus]
\begin{center}
\begin{tikzpicture}
\end{tikzpicture}
\end{center}
\begin{hint}
\end{hint}
\begin{sol}
\end{sol}
\end{exo}


\Closesolutionfile{indications}
\Closesolutionfile{solutions}

\indications
\correction

\end{document}

