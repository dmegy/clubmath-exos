\documentclass[11pt,a4paper]{article}
\usepackage[french]{babel}
\usepackage[utf8]{inputenc}
\usepackage{mathtools,amssymb,amsthm}
\usepackage{mathrsfs,stmaryrd}
\usepackage{fancybox,mdframed,multicol,comment,enumitem}
\usepackage{microtype}

\usepackage{hyperref}
\hypersetup{
    colorlinks=true,       % false: boxed links; true: colored links
    linkcolor=blue,          % color of internal links
    citecolor=blue,        % color of links to bibliography
    filecolor=blue,      % color of file links
    urlcolor=black           % color of external links
}

%\usepackage[dvipsnames]{xcolor}

\usepackage[normalem]{ulem} % pour souligner avec changements de ligne
\usepackage{pgf,pgfmath,tikz}
\usetikzlibrary{arrows}
\usetikzlibrary[patterns]
\tikzset{every picture/.style={execute at begin picture={
   \shorthandoff{:;!?};}
}}

\newcommand*\circled[1]{\tikz[baseline=(char.base)]{
            \node[shape=circle,draw,inner sep=2pt] (char) {#1};}}


\def\point{node {$\bullet$}}

\usepackage{tkz-euclide}



\theoremstyle{definition}
\newtheorem{theoreme}{Théorème}[section]
\newtheorem{definition}[theoreme]{Définition}
\newtheorem{definitions}[theoreme]{Définitions}
\newtheorem{lemme}[theoreme]{Lemme}
\newtheorem{proposition}[theoreme]{Proposition}
\newtheorem{corollaire}[theoreme]{Corollaire}
\newtheorem{remarque}[theoreme]{Remarque}
\newtheorem{ex}{Problème}

\newcommand{\N}{\mathbb N}
\newcommand{\Z}{\mathbb Z}
\newcommand{\Q}{\mathbb Q}
\newcommand{\R}{\mathbb R}
\newcommand{\C}{\mathbb C}
\newcommand{\U}{\mathbb U}
\newcommand{\F}{\mathbb F}
\newcommand{\G}{\mathbb G}

\newcommand*{\etoile}
{
\begin{center}
$\star$\par
$\star$\hspace*{3ex}$\star$
\end{center}
}

\newcommand{\ensemble}[2]{\left \{ #1  
    \ifx&#2&%
       %
    \else%
       \, \middle | \, #2%
    \fi%
\right \}}

\newcommand{\modulo}[1]{\:\left(\operatorname{mod}\:#1\right)}

% délimiteurs

\DeclarePairedDelimiter{\abs}{\lvert}{\rvert}
\DeclarePairedDelimiter{\ceil}{\lceil}{\rceil}
\DeclarePairedDelimiter{\floor}{\lfloor}{\rfloor}


%%%%%%%%%%%%%%%%%%%%%%%%%%%%%%%%%
%%%%%% MISE EN FORME CLUB %%%%%%%
%%%%%%%%%%%%%%%%%%%%%%%%%%%%%%%%%

\pagestyle{empty}

\usepackage[margin=2.5cm]{geometry}
\everymath{\displaystyle}
\usepackage{fourier}
%\usepackage{fourier,eulervm}% Adobe Utopia et Euler
% En-tête des feuilles :

\newcommand{\enTete}[1]{
\noindent \textbf{\textsf{\href{http://depmath-nancy.univ-lorraine.fr/club/}{Club Mathématique de Nancy} \hfill Institut Élie Cartan}}
\hrule
\begin{center}
{\Huge \textbf{#1}}
\end{center}
\hrule
\vspace{1em}
}

\newcommand{\avertissement}{\begin{mdframed}[linewidth=1pt]\textbf{AVERTISSEMENT ! Ce document est un brouillon qui sert de catalogue pour les feuilles d'exos du club mathématique de Nancy \url{https://dmegy.perso.math.cnrs.fr/club/}. Ne pas diffuser tel quel aux élèves ni de façon large sur le net, il reste des coquilles et énoncés parfois peu précis. Ce document a vocation a rester inachevé. Il peut néanmoins être utile aux enseignants. Enfin, ce document change en permanence, la version à jour est récupérable sur \url{https://github.com/dmegy/clubmath-exos}.}\end{mdframed}}




% - - - - - - - - - - - - - -
% PARAMETRAGE DU PACKAGE ANSWERS 
% POUR LES INDICATIONS ET CORRECTIONS
% - - - - - - - - - - - - - - 

\usepackage{answers}

\Newassociation{sol}{Soln}{solutions}
% ira dans le fichier d'identifiant 'solutions'
% et écrira les solutions dans un environnement 'Soln'
\Newassociation{hint}{Hint}{indications}

\newenvironment{exo}{\begin{ex} \label{enonce.\theex} }{\end{ex} }

\renewenvironment{Soln}[1]{\noindent{\bf Correction de l'exercice \ref{enonce.#1}.} \\ }

\renewenvironment{Hint}[1]{ \noindent{\bf Exercice  \ref{enonce.#1}.} \label{hint.#1}}


% - - - - - - - - - - - - - - 
% FIN PARAMETRAGE ANSWERS
% - - - - - - - - - - - - - - 

%-----------------------------
% MACROS POUR LES FEUILLES DE TD


\newenvironment{feuilleTD}{


\Opensolutionfile{indications}[\jobname_hints]
\Opensolutionfile{solutions}[\jobname_sol]
}{
\Closesolutionfile{indications}
\Closesolutionfile{solutions}
}

\newcommand{\indications}{
\newpage
\noindent {\Large \bf Indications} \hrulefill

\vspace{1em}
\Readsolutionfile{indications}
}

\newcommand{\correction}{
\newpage
\hrule
\begin{center}
{\Large \bf Correction}
\end{center}
\hrule
\vspace{1em}
\Readsolutionfile{solutions}
}



\begin{document}
\Opensolutionfile{indications}[_\jobname_hints]
\Opensolutionfile{solutions}[_\jobname_sol]


\title{Autour du théorème de Thalès}
\author{Damien Mégy}
\maketitle

\avertissement 


\tableofcontents

\section{Uniquement avec le théorème des milieux}

Remettre les exos du bouquin déjà tapés

% - - - - - - - - - -
\begin{exo}
Énoncé
\begin{hint}
Indication.
\end{hint}
\begin{sol}
Correction.
\end{sol}
\end{exo}

% - - - - - - - - - -
\begin{exo}
Énoncé
\begin{hint}
Indication.
\end{hint}
\begin{sol}
Correction.
\end{sol}
\end{exo}


Petit exo marrant:

\url{https://www.youtube.com/watch?v=wIk97NRwj5I}


\section{Thalès, version générale mais non croisé}

TODO : rajouter \url{https://www.youtube.com/watch?v=l6XaE_U9FDU} Thalès, mais aussi triangles semblables. Déplacer dans triangles semblables ?

% - - - - - - - - - -
\begin{exo}
Énoncé
\begin{hint}
Indication.
\end{hint}
\begin{sol}
Correction.
\end{sol}
\end{exo}

% - - - - - - - - - -
\begin{exo}
Énoncé
\begin{hint}
Indication.
\end{hint}
\begin{sol}
Correction.
\end{sol}
\end{exo}

\section{Thalès, version générale avec croisements}

% - - - - - - - - - -
\begin{exo}
Énoncé
\begin{hint}
Indication.
\end{hint}
\begin{sol}
Correction.
\end{sol}
\end{exo}

% - - - - - - - - - -
\begin{exo}
Énoncé
\begin{hint}
Indication.
\end{hint}
\begin{sol}
Correction.
\end{sol}
\end{exo}


% - - - - - - - -
% - - - - - - - -
% - - - - - - - -
%\begin{exo}[Théorème de Varignon] \label{Varignon}
%Soit $ABCD$ un quadrilatère convexe quelconque et $I$, $J$, $K$, $L$ les milieux de ses côtés.
%Montrer que $IJKL$ est ... un parallélogramme. (Toujours!)
%%Montrer ensuite que l'aire de $ABCD$ est le double de celle de $IJKL$.
%\begin{center}
%\begin{tikzpicture}
%\tkzDefPoints{-4/-1/A, 4/-1.5/B, 3/1.5/C, -2/1/D, 0/0/O}
%\tkzDefMidPoint(A,B)\tkzGetPoint{I}
%\tkzDefMidPoint(B,C)\tkzGetPoint{J}
%\tkzDefMidPoint(C,D)\tkzGetPoint{K}
%\tkzDefMidPoint(D,A)\tkzGetPoint{L}
%
%
%\tkzDrawPolygon[very thick](A,B,C,D)
%\tkzDrawPolygon[dashed](I,J,K,L)
%\tkzDrawPoints(I,J,K,L)
%\tkzMarkSegments[mark=|||](A,I I,B)
%\tkzMarkSegments[mark=||](B,J J,C)
%\tkzMarkSegments[mark=s](C,K K,D)
%\tkzMarkSegments[mark=|](D,L L,A)
%\tkzAutoLabelPoints[center=O](A,B,C,D,I,J,K,L)
%\end{tikzpicture}
%\end{center}
%\begin{hint} 
%Tracer les diagonales $[AC]$ et $[BD]$. % et utiliser Thalès
%\end{hint}
%% couper le quadrilatère en deux triangles et montrer que les côtés sont parallèles à l'aide de Thalès.
%% pour une deuxième preuve de l'aire, voir 
%% http://serge.mehl.free.fr/anx/th_varignon.html
%\begin{sol}
%\begin{enumerate}
%\item 
%Dans le triangle $ABC$, en notant $I$ est le milieu de $[AB]$ et $J$ le milieu de $[BC]$, le théorème de Thalès dit que $(IJ)$ est parall\`ele \`a $(AC)$ et $IJ = \frac{1}{2} AC$. On raisonne pareillement avec le triangle $ACD$, ce qui donne $(KL)$ parall\`ele \`a $AC$ et $KL = \frac{1}{2} AC$. Or, un quadrilat\`ere qui a deux c\^ot\'es parall\`eles et de m\^eme longueur est un parall\'elogramme.
%
%\item La preuve la plus élémentaire utilise uniquement qu'une médiane d'un triangle donné le partage en deux triangles de même aire.
%
%Soit $O$ le point d'intersection des diagonales du quadrilat\`ere $ABCD$. On consid\`ere le triangle $AOB$. Soit $O_1$ le point d'intersection de la diagonale $[AC]$ avec $[IL]$ et soit $O_2$ le point d'intersection de $[IJ]$ avec la diagonale $[BD]$.
%
%% Par construction, le quadrilat\`ere $IO_1OO_2$ est un parall\'elogramme. 
% 
%Par le théorème de Thalès,  $O_1$ est le milieu de $[AO]$ et $O_2$ le milieu de $[BO]$. Les triangles $IO_1A$ et $IO_1B$ ont même aire, de même que les triangles $I0_2O$ et $IO_2B$.
%
%La somme des aires des triangles $AIO_1$ et $IO_2B$ est donc exactement \'egale \`a l'aire du parallélogramme $IO_1OO_2$. 
% 
%On applique le même raisonnement aux triangles $BCO$, $CDO$ et $ADO$, ce qui signifie que, dans le quadrilatère $ABCD$, la partie compl\'ementaire de $IJKL$ a une aire qui est exactement \'egale \`a celle de $IJKL$, ce qui permet de conclure. \end{enumerate}
%
%\end{sol}
%\end{exo}
%
%- - - - - 



\begin{exo}[Centre de gravité d'un quadrilatère]
Soit $ABCD$ un quadrilatère et $I$, $J$, $K$ et $L$ les milieux de ses côtés.
Montrer que le milieu du segment $[IK]$ est aussi le milieu du segment $[JL]$.

(Ce point est appelé le \emph{centre de gravité} des quatre sommets, ou aussi l'\emph{isobarycentre} des quatre sommets. Attention, ce n'est \underline{pas} le centre de gravité du quadrilatère plein, contrairement au cas des triangles ! Ce n'est pas non plus l'intersection des diagonales.
%-> exo sur le centre de gravité du quadrilatère plein ?
)
% http://mathafou.free.fr/pbw/pb427.html
\begin{center}
\def\isobarycentre{
\tkzDefPoints{0/0/A, 5/0/B, 4/4/C, 1/3/D}
\tkzDefMidPoint(A,B)\tkzGetPoint{I}
\tkzDefMidPoint(B,C)\tkzGetPoint{J}
\tkzDefMidPoint(C,D)\tkzGetPoint{K}
\tkzDefMidPoint(D,A)\tkzGetPoint{L}
\tkzDefMidPoint(I,K)\tkzGetPoint{O}
\tkzDrawPolygon(A,B,C,D)
\tkzDrawPoints(A,B,C,D,I,J,K,L,O)
\tkzMarkSegments[mark=|||](A,I I,B)
\tkzMarkSegments[mark=||](B,J J,C)
\tkzMarkSegments[mark=|](C,K K,D)
\tkzMarkSegments[mark=s](D,L L,A)
\tkzAutoLabelPoints[center=O](A,B,C,D,I,J,K,L)
}
\begin{tikzpicture}[rotate=10,scale=.7]
\isobarycentre{}
\tkzDrawSegment[dashed](I,K)
\tkzMarkSegments[mark=x](I,O O,K)
\end{tikzpicture}
\hspace{1cm}
\begin{tikzpicture}[rotate=10,scale=.7]
\isobarycentre{}
\tkzDrawSegment[dashed](J,L)
\tkzMarkSegments[mark=o](J,O O,L)
\end{tikzpicture}
\end{center}
\end{exo}





% - - - - - - - -
% - - - - - - - -
\begin{exo}%vecten
Sur l'extérieur d'un triangle $ABC$, on accole des carrés $ABIJ$ et $CAMN$ de centres $Z$ et $Y$.
On note $X$, le milieu du segment $[BC]$.
Montrer que le triangle $XYZ$ est rectangle isocèle en $X$.
\begin{center}
\begin{tikzpicture}[scale=.6,rotate=90]
\tkzDefPoint(190:2){A}
\tkzDefPoint(60:2){B}
\tkzDefPoint(-10:2){C}
\tkzDefPoint(0,0){O}
\tkzDefSquare(A,B) \tkzGetFirstPoint{I}\tkzGetSecondPoint{J}
\tkzDefSquare(B,C) \tkzGetFirstPoint{K}\tkzGetSecondPoint{L}
\tkzDefSquare(C,A) \tkzGetFirstPoint{M}\tkzGetSecondPoint{N}
%\tkzDrawPolygon(A,B,C)
\tkzDrawPolygon(A,B,I,J)
\tkzDrawPolygon(C,A,M,N)
\tkzDrawPolygon[very thick](A,B,C)

\tkzDefMidPoint(M,C)\tkzGetPoint{Y}
\tkzDefMidPoint(C,B)\tkzGetPoint{X}
\tkzDefMidPoint(B,J)\tkzGetPoint{Z}
\tkzDrawPoints(X,Y,Z)
\tkzDrawPolygon[dashed](X,Y,Z)
\tkzAutoLabelPoints[center=O](A,B,C,I,J,M,N,X,Y,Z)
\end{tikzpicture}
\end{center}
\begin{hint}
Considérer une rotation de centre $A$ et d'angle $90^\circ$.
\end{hint}
\begin{sol}
La rotation de centre $A$ et d'angle $90^\circ$ envoie $B$ sur $J$ et $M$ sur $C$.
Elle envoie donc le segment $[BM]$ sur $[JC]$.
On en déduit que ces deux segments sont de même longueur et perpendiculaires.

Ensuite on applique Thalès pour en déduire que $[XY]$ et $[XZ]$ sont de même longueur et perpendiculaires.
\end{sol}
\end{exo}

% - - - - - - 


% - - - - - - - -
% - - - - - - - -
% - - - - - - - -
\begin{exo}[Partage en trois]
%barycentres
Soit $ABCD$ un parallélogramme, $M$ le milieu de $[AB]$ et $N$ le milieu de $[CD]$. Montrer que les droites $(DM)$ et $(BN)$ coupent la diagonale $[AC]$ en deux points $K$ et $L$ qui la partagent en trois segments égaux.
\begin{center}
\begin{tikzpicture}[rotate=-5]
\tkzDefPoints{-3/-1/A, 2/-1/B, 3/1/C, -2/1/D, 0/0/O}
\tkzDefMidPoint(A,B)\tkzGetPoint{M}
\tkzDefMidPoint(C,D)\tkzGetPoint{N}
\tkzInterLL(D,M)(A,C)\tkzGetPoint{K}
\tkzInterLL(B,N)(A,C)\tkzGetPoint{L}
\tkzDrawPolygon(A,B,C,D)
\tkzDrawSegments[dashed](D,M B,N A,C)
\tkzMarkSegments[mark=||](A,M M,B C,N N,D)
\tkzLabelPoints[above=3pt](D,C,K,N)
\tkzLabelPoints[below=3pt](A,B,L,M)
\tkzDrawPoints(M,N,K,L)
\end{tikzpicture}
\end{center}

\begin{hint}
Il y a plusieurs triangles dans lesquels on peut appliquer le théorème de Thalès.
\end{hint}
\begin{sol}
On a $MB=DN$ donc $MBND$ est un parallélogramme.

On en déduit que $(DM)//(BN)$.

Comme $M$ est le milieu de $[AB]$, on a par le théorème des milieux que $AK=KL$.

De la même façon, le théorème de Thalès appliqué dans $DCK$ entraîne que $KL=LC$, d'où le résultat.

\underline{Variante, demande plus d'initiative} :\\
Soit $Q$ le symétrique de $M$ par rapport à $B$. On a $MB=BQ$, donc $BQ=NC$ et $(BQ)//(NC)$. Donc $BQCM$ est un parallélogramme et donc $(NB)//(CQ)$.

Comme $AM=MB=BQ$ et que $(MK)//(BL)//(QC)$, le théorème de Thalès donne $AK=KL=LC$.

\underline{Autre preuve, avec centre de gravité} :\\
Dans le triangle $ABD$, le point $K$ est l'intersection des deux médianes $(DM)$ et $(AO)$. C'est donc le centre de gravité de $ABD$.

On en déduit que $KA = 2KO$.

Par symétrie centrale de centre $O$, on a $AK=CL$ et $KO=LO$, et finalement $AK = KO+OL = KL = LC$.
\end{sol}
\end{exo}




% - - - - - - - -
% - - - - - - - -
% - - - - - - - -
\begin{exo}[Médiane sur médiane]
% source : http://ressources.unisciel.fr/sillages/mathematiques/geometrie/res/Barycentres.pdf
% faisable avec barycentres

Soit $ABC$ un triangle, $M$ le milieu de $[BC]$ et $N$ le milieu de $[AM]$. À quel endroit la droite $(CN)$ coupe-t-elle le segment $[AB]$ ?
\begin{center}
\begin{tikzpicture}
\tkzDefPoints{1/4/A,-3/0/B, 3/0/C}
\tkzDefMidPoint(B,C)\tkzGetPoint{M}
\tkzDefMidPoint(A,M)\tkzGetPoint{N}
\tkzDrawPolygon(A,B,C)
\tkzDrawSegment(A,M)
\tkzMarkSegments[mark=||](B,M M,C)
\tkzMarkSegments[mark=|](M,N N,A)

\tkzLabelPoints[below=3pt](B,M,C)
\tkzLabelPoints[above=3pt](A)
\tkzLabelPoints[above right](N)
\tkzInterLL(C,N)(A,B)\tkzGetPoint{X}
\tkzDrawPoints(M,N)

\tkzDrawLine[dashed](C,X)
\tkzDrawPoint[fill=white](X)
\tkzLabelPoint[above=3pt](X){$?$}
\end{tikzpicture}
\end{center}

\begin{hint}
Tracer la parallèle à $(PC)$ passant par $M$.
\end{hint}
\begin{sol}
La parallèle à $(PC)$ passant par $M$ coupe le segment $[AB]$ en un point $Q$. D'après le théorème de Thalès dans le triangle $AQM$, on a : 
\[ \frac{AP}{AQ}=\frac{AN}{AM}.\]
On en déduit que  $P$ est le milieu du segment $[AQ]$.

Appliquons maintenant le théorème de Thalès dans le triangle $BPC$ : on obtient alors
\[\frac{BM}{BC}=\frac{BQ}{BP}, \]
et donc $Q$ est le milieu du segment $[BP]$.

On en déduit donc que $BQ=QP=PA$, et donc la droite $(CN)$ coupe le segment $[AB]$ aux deux tiers.
\end{sol}
\end{exo}




% - - - - - - - -
% - - - - - - - -
\begin{exo}[Échange médiane contre hauteur]%vecten, moulin à vent
Sur l'extérieur d'un triangle $ABC$, on accole des carrés $ABIJ$ et $BCKL$.
On note $P$ le milieu du segment$[IL]$.
Montrer que la droite $(PB)$ est perpendiculaire à $(AC)$ et que de plus $AC=2PB$.
\begin{center}
\begin{tikzpicture}[scale=.7]
\tkzDefPoint(0,0){A}
\tkzDefPoint(1,3){B}
\tkzDefPoint(4,0){C}
\tkzDefPoint(0,0){O}
\tkzDefSquare(A,B) \tkzGetFirstPoint{I}\tkzGetSecondPoint{J}
\tkzDefSquare(B,C) \tkzGetFirstPoint{K}\tkzGetSecondPoint{L}
%\tkzDrawPolygon(A,B,C)
\tkzDrawPolygon(A,B,I,J)
\tkzDrawPolygon(B,C,K,L)

\tkzDefMidPoint(I,L)\tkzGetPoint{P}
\tkzDrawSegment(I,P)
\tkzDrawSegment(P,L)
\tkzMarkSegment[mark=s||](P,L)
\tkzMarkSegment[mark=s||](I,P)
\tkzDrawSegment(A,C)
\tkzDrawSegment[dashed](P,B)
\tkzAutoLabelPoints[center=B](A,C,I,J,K,L,P)
\tkzLabelPoints[left](B)
\end{tikzpicture}
\end{center}
\begin{hint}
Considérer une rotation de $90^\circ$ de centre $B$, appliquée au triangle $ABC$.
\end{hint}
\begin{sol}
ATTENTIOn finir figure.


La rotation d'angle $90^\circ$ et de centre $B$ envoie le triangle $ABC$ sur le triangle $IBM$, où $M$ est le symétrique de $L$ par rapport à $B$.

On en déduit que $IM=AC$ et que $(IM)\bot(AC)$.
Maintenant, dans le triangle $ILM$, on applique le théorème des milieux avec le segment $[PB]$.

On aurait pu appliquer la rotation à $IBL$, ça marche aussi.
On peut aussi considérer la rotation de centre $B$ et d'angle $-90^\circ$, ça marche aussi.
\end{sol}
\end{exo}


%%%%%%%%%%%%%%%%%%%%%%%%%%%%
\begin{exo}[Deux triangles isocèles rectangles] % exo7
% rotations
Soient $AOB$ et $COD$ deux triangles directs, isocèles rectangles en $O$. Soient $I$, $J$, $K$ et $L$ les milieux des segments $[AB]$, $[BC]$, $[CD]$ et $[DA]$.
%Montrer que $(AC)\bot (BD)$ et que $AC=BD$.
Montrer que $IJKL$ est un carré. 
\begin{center}
\definecolor{uuuuuu}{rgb}{0,0,0}
\definecolor{qqqqff}{rgb}{0.,0.,1.}
\begin{tikzpicture}[line cap=round,line join=round,>=triangle 45,scale=.9]
\clip(-4.42,-2.62) rectangle (7.18,6.76);
\draw[very thick] (-3.5,1.02)-- (0.48,1.46) -- (0.04,5.44) -- (-3.5,1.02);
\draw[very thick] (2.2,0.)-- (0.48,1.46) -- (1.94,3.18) -- (2.2,0.);
%\draw [dash pattern=on 2pt off 2pt] (-3.5,1.02)-- (1.94,3.18); % indication
%\draw [dash pattern=on 2pt off 2pt] (2.2,0.)-- (0.04,5.44);
\draw (-3.5,1.02)-- (2.2,0.);
\draw (1.94,3.18)-- (0.04,5.44);
\draw (-0.65,0.51)-- (-1.73,3.23);
\draw (-1.73,3.23)-- (0.99,4.31);
\draw (0.99,4.31)-- (2.07,1.59);
\draw (2.07,1.59)-- (-0.65,0.51);
\begin{scriptsize}
\draw [fill=uuuuuu] (0.48,1.46) circle (1.5pt);
\draw[color=uuuuuu] (0.2,1.15) node {$O$};
\draw [fill=uuuuuu] (-3.5,1.02) circle (1.5pt);
\draw[color=uuuuuu] (-3.68,1.51) node {$A$};
\draw [fill=uuuuuu] (0.04,5.44) circle (1.5pt);
\draw[color=uuuuuu] (0.18,5.81) node {$B$};
\draw [fill=uuuuuu] (1.94,3.18) circle (1.5pt);
\draw[color=uuuuuu] (2.08,3.55) node {$C$};
\draw [fill=uuuuuu] (2.2,0.) circle (1.5pt);
\draw[color=uuuuuu] (2.34,0.37) node {$D$};
\draw [fill=uuuuuu] (-1.73,3.23) circle (1.5pt);
\draw[color=uuuuuu] (-2.34,3.33) node {$I$};
\draw [fill=uuuuuu] (0.99,4.31) circle (1.5pt);
\draw[color=uuuuuu] (1.22,4.69) node {$J$};
\draw [fill=uuuuuu] (2.07,1.59) circle (1.5pt);
\draw[color=uuuuuu] (2.3,1.79) node {$K$};
\draw [fill=uuuuuu] (-0.65,0.51) circle (1.5pt);
\draw[color=uuuuuu] (-0.8,0.03) node {$L$};
\end{scriptsize}
\end{tikzpicture}
\end{center}


\begin{hint}
Tracer les \og diagonales\fg{} $[AC]$ et $[BD]$.
\end{hint}
\begin{sol}
ATTENTION VIRER LES VECTEURS pour le collège.
On commence par prouver que $(AC)\bot (BD)$ et que $AC=BD$. Ensuite on rouve le résultat.

\begin{enumerate}
\item Soit $\rho$ la rotation de centre $O$ et d'angle $\pi/2$. D'après l'énoncé, on a $\rho(B)=A$ et $\rho(D)=C$. Donc $[AC]$ est l'image de $[AD]$ par $\rho$, d'où on déduit que $(AC)\bot (BD)$ et que $AC=BD$.

\item Le quadrilatère $IJKL$ est toujours un parallélogramme, même sans hypothèses sur $ABCD$ (c'est le théorème de Varignon). En effet, par le théorème de Thalès (ou simplement le théorème des milieux) : 
\[ \overrightarrow{IL} = \frac12 \overrightarrow{BD} = \overrightarrow{JK}.\]
Ceci signifie que les côtés $[IL]$ et $[JK]$ ont même longueur et sont parallèles, donc $IJKL$ est un parallélogramme.

Pour voir que c'est un carré, remarquons qu'on montre de même que  
\[ \overrightarrow{IJ} = \frac12 \overrightarrow{AC} = \overrightarrow{LK},\]
et d'après la première question, $(AC)\bot (BD)$ et que $AC=BD$, donc $IJKL$ est un parallélogramme ayant un angle droit et des cotés consécutifs de même longueur. C'est donc un carré.
\end{enumerate}
\end{sol}
\end{exo}

% - - - - ---  - 
% - - - - - - - -
% - - - - - - - -
\begin{exo}[Théorème de la bissectrice]

Soit $ABC$ un triangle, et $M$ le pied de la bissectrice issue de $A$. Montrer que $\frac{AB}{AC}=\frac{MB}{MC}$.
\begin{center}
\begin{tikzpicture}[scale=.8]
\tkzDefPoints{1/4/A, 0/0/B, 6/0/C}
\tkzDrawPolygon[very thick](A,B,C)
\tkzDefLine[bisector](B,A,C) \tkzGetPoint{x}
\tkzInterLL(A,x)(B,C)\tkzGetPoint{M}
\tkzDrawSegment[dashed](A,M)
\tkzDrawPoint[fill=white](M)
\tkzMarkAngles[mark=|](B,A,M M,A,C)
\tkzLabelPoints[above](A)
\tkzLabelPoints[below](B,M,C)
\end{tikzpicture}
\end{center}
\begin{hint}
Tracer la parallèle à la bissectrice passant par un des sommets.
\end{hint}
\begin{sol}
\end{sol}
\end{exo}
%\newpage
% - - - - - - - -
% - - - - - - - -
\begin{exo}[Le miraculeux \og cercle des neuf points\fg]
\def\neufpoints{
\tkzDefPoints{0/0/A,6/0/B,1/4/C}

\tkzDefTriangleCenter[centroid](A,B,C)
\tkzGetPoint{G}
\tkzDefSpcTriangle[medial](A,B,C){I,J,K}

\tkzDefSpcTriangle[orthic](A,B,C){H_A,H_B,H_C}
\tkzDefTriangleCenter[ortho](B,C,A)
\tkzGetPoint{H}

\tkzDrawSegments[thin](A,H_A B,H_B C,H_C)
\tkzMarkRightAngles[fill=gray!20,
opacity=.5](A,H_A,C B,H_B,A C,H_C,A)

\tkzDefMidPoint(A,H)\tkzGetPoint{X}
\tkzDefMidPoint(B,H)\tkzGetPoint{Y}
\tkzDefMidPoint(C,H)\tkzGetPoint{Z}
\tkzMarkSegments[size=1.5pt,mark=|](A,X X,H)
\tkzMarkSegments[size=1.5pt,mark=|||](B,Y Y,H)
\tkzMarkSegments[size=1.5pt,mark=||](C,Z Z,H)
\tkzDrawPolygon(A,B,C)
\tkzDrawPoints[size=5pt](I,J,K,H_A,H_B,H_C,X,Y,Z)
\tkzDrawPoints[fill=white,draw=black](H)

}
Soit $ABC$ un triangle.
On place:
\begin{itemize}
\item les milieux des trois côtés $I$, $J$ et $K$;
\item les pieds des hauteurs $H_A$ $H_B$ et $H_C$;
\item les points $X$, $Y$ et $Z$ situés à mi-chemin entre l'orthocentre $H$ et les sommets du triangle.
\end{itemize}
\begin{center}
\begin{tikzpicture}[scale=1.3]
\neufpoints{}
\tkzAutoLabelPoints[center=G](A,B,C,I,J,K)
\tkzAutoLabelPoints[center=H](H_A,H_C)
\tkzLabelPoints[left=3pt](H_B)
\tkzLabelPoints[below](X,Y)
\tkzLabelPoints[right](Z)
\end{tikzpicture}
\end{center}
Le théorème des neuf cercles est la propriété incroyable suivante : 
\begin{mdframed}
\noindent \textbf{Théorème} : les neuf points $I$, $J$, $K$, $H_A$, $H_B$, $H_C$, $X$, $Y$ et $Z$ sont sur un seul et même cercle.
\end{mdframed}
Ce cercle est appelé cercle des neuf points, cercle d'Euler, cercle de Feuerbach ou encore cercle de Terquem.

Pour démontrer le théorème, on s'intéressera au quadrilatère $IJXY$, puis à d'autres du même type. Ne pas hésiter à demander des indications, ce problème n'est pas facile.
\begin{hint}
Ce magnifique théorème se démontre en plusieurs étapes.

\paragraph{Étape 1}
Montrer que le quadrilatère $IJXY$ est un parallélogramme en appliquant le théorème de Thalès dans deux triangles différents.
\begin{center}
\begin{tikzpicture}[scale=1.3]
\tkzDefPoints{0/0/A,6/0/B,1/4/C}

\tkzDefTriangleCenter[centroid](A,B,C)
\tkzGetPoint{G}
\tkzDefSpcTriangle[medial](A,B,C){I,J,K}

\tkzDefSpcTriangle[orthic](A,B,C){H_A,H_B,H_C}
\tkzDefTriangleCenter[ortho](B,C,A)
\tkzGetPoint{H}

\tkzDrawSegments[thin](A,H_A B,H_B C,H_C)
\tkzMarkRightAngles[fill=gray!20,
opacity=.5](A,H_A,C B,H_B,A C,H_C,A)

\tkzDefMidPoint(A,H)\tkzGetPoint{X}
\tkzDefMidPoint(B,H)\tkzGetPoint{Y}
\tkzDefMidPoint(C,H)\tkzGetPoint{Z}
\tkzMarkSegments[size=1.5pt,mark=|](A,X X,H)
\tkzMarkSegments[size=1.5pt,mark=|||](B,Y Y,H)
\tkzMarkSegments[size=1.5pt,mark=||](C,Z Z,H)
\tkzDrawPolygon(A,B,C)
\tkzDrawPoints[size=5pt](I,J,K,H_A,H_B,H_C,X,Y,Z)
\tkzDrawPoints[fill=white,draw=black](H)
\tkzAutoLabelPoints[center=G](A,B,C,I,J,K)
\tkzAutoLabelPoints[center=H](H_A,H_C)
\tkzLabelPoints[left=3pt](H_B)
\tkzDrawPolygon[dashed](I,J,X,Y)% rectangle
\tkzLabelPoints[below](X,Y)
\tkzLabelPoints[right](Z)
\end{tikzpicture}
\end{center}
\paragraph{Étape 1bis}
Montrer que le quadrilatère $IJXY$ est en réalité un rectangle et en déduire que $I$, $J$, $X$ et $Y$ se trouvent sur un même cercle $\mathcal C$.
\paragraph{Étape 2}
Montrer que les pieds des hauteurs $H_A$, et $H_B$ se trouvent également sur ce cercle $\mathcal C$. À ce stade il ne manque plus que trois points.
\paragraph{Étape 3}
En appliquant les étapes 1 et 1bis à un autre quadrilatère, en déduire que $Z$ et $K$ sont eux aussi sur le cercle $\mathcal C$.
\paragraph{Étape 4 et fin}
Comme dans l'étape 2, montrer que $H_C$ est sur le cercle.
\end{hint}
\end{exo}






\Closesolutionfile{indications}
\Closesolutionfile{solutions}

\indications
\correction

\end{document}

