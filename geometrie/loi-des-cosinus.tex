\documentclass[11pt,a4paper]{article}
\usepackage[french]{babel}
\usepackage[utf8]{inputenc}
\usepackage{mathtools,amssymb,amsthm}
\usepackage{mathrsfs,stmaryrd}
\usepackage{fancybox,mdframed,multicol,comment,enumitem}
\usepackage{microtype}

\usepackage{hyperref}
\hypersetup{
    colorlinks=true,       % false: boxed links; true: colored links
    linkcolor=blue,          % color of internal links
    citecolor=blue,        % color of links to bibliography
    filecolor=blue,      % color of file links
    urlcolor=black           % color of external links
}

%\usepackage[dvipsnames]{xcolor}

\usepackage[normalem]{ulem} % pour souligner avec changements de ligne
\usepackage{pgf,pgfmath,tikz}
\usetikzlibrary{arrows}
\usetikzlibrary[patterns]
\tikzset{every picture/.style={execute at begin picture={
   \shorthandoff{:;!?};}
}}

\newcommand*\circled[1]{\tikz[baseline=(char.base)]{
            \node[shape=circle,draw,inner sep=2pt] (char) {#1};}}


\def\point{node {$\bullet$}}

\usepackage{tkz-euclide}



\theoremstyle{definition}
\newtheorem{theoreme}{Théorème}[section]
\newtheorem{definition}[theoreme]{Définition}
\newtheorem{definitions}[theoreme]{Définitions}
\newtheorem{lemme}[theoreme]{Lemme}
\newtheorem{proposition}[theoreme]{Proposition}
\newtheorem{corollaire}[theoreme]{Corollaire}
\newtheorem{remarque}[theoreme]{Remarque}
\newtheorem{ex}{Problème}

\newcommand{\N}{\mathbb N}
\newcommand{\Z}{\mathbb Z}
\newcommand{\Q}{\mathbb Q}
\newcommand{\R}{\mathbb R}
\newcommand{\C}{\mathbb C}
\newcommand{\U}{\mathbb U}
\newcommand{\F}{\mathbb F}
\newcommand{\G}{\mathbb G}

\newcommand*{\etoile}
{
\begin{center}
$\star$\par
$\star$\hspace*{3ex}$\star$
\end{center}
}

\newcommand{\ensemble}[2]{\left \{ #1  
    \ifx&#2&%
       %
    \else%
       \, \middle | \, #2%
    \fi%
\right \}}

\newcommand{\modulo}[1]{\:\left(\operatorname{mod}\:#1\right)}

% délimiteurs

\DeclarePairedDelimiter{\abs}{\lvert}{\rvert}
\DeclarePairedDelimiter{\ceil}{\lceil}{\rceil}
\DeclarePairedDelimiter{\floor}{\lfloor}{\rfloor}


%%%%%%%%%%%%%%%%%%%%%%%%%%%%%%%%%
%%%%%% MISE EN FORME CLUB %%%%%%%
%%%%%%%%%%%%%%%%%%%%%%%%%%%%%%%%%

\pagestyle{empty}

\usepackage[margin=2.5cm]{geometry}
\everymath{\displaystyle}
\usepackage{fourier}
%\usepackage{fourier,eulervm}% Adobe Utopia et Euler
% En-tête des feuilles :

\newcommand{\enTete}[1]{
\noindent \textbf{\textsf{\href{http://depmath-nancy.univ-lorraine.fr/club/}{Club Mathématique de Nancy} \hfill Institut Élie Cartan}}
\hrule
\begin{center}
{\Huge \textbf{#1}}
\end{center}
\hrule
\vspace{1em}
}

\newcommand{\avertissement}{\begin{mdframed}[linewidth=1pt]\textbf{AVERTISSEMENT ! Ce document est un brouillon qui sert de catalogue pour les feuilles d'exos du club mathématique de Nancy \url{https://dmegy.perso.math.cnrs.fr/club/}. Ne pas diffuser tel quel aux élèves ni de façon large sur le net, il reste des coquilles et énoncés parfois peu précis. Ce document a vocation a rester inachevé. Il peut néanmoins être utile aux enseignants. Enfin, ce document change en permanence, la version à jour est récupérable sur \url{https://github.com/dmegy/clubmath-exos}.}\end{mdframed}}




% - - - - - - - - - - - - - -
% PARAMETRAGE DU PACKAGE ANSWERS 
% POUR LES INDICATIONS ET CORRECTIONS
% - - - - - - - - - - - - - - 

\usepackage{answers}

\Newassociation{sol}{Soln}{solutions}
% ira dans le fichier d'identifiant 'solutions'
% et écrira les solutions dans un environnement 'Soln'
\Newassociation{hint}{Hint}{indications}

\newenvironment{exo}{\begin{ex} \label{enonce.\theex} }{\end{ex} }

\renewenvironment{Soln}[1]{\noindent{\bf Correction de l'exercice \ref{enonce.#1}.} \\ }

\renewenvironment{Hint}[1]{ \noindent{\bf Exercice  \ref{enonce.#1}.} \label{hint.#1}}


% - - - - - - - - - - - - - - 
% FIN PARAMETRAGE ANSWERS
% - - - - - - - - - - - - - - 

%-----------------------------
% MACROS POUR LES FEUILLES DE TD


\newenvironment{feuilleTD}{


\Opensolutionfile{indications}[\jobname_hints]
\Opensolutionfile{solutions}[\jobname_sol]
}{
\Closesolutionfile{indications}
\Closesolutionfile{solutions}
}

\newcommand{\indications}{
\newpage
\noindent {\Large \bf Indications} \hrulefill

\vspace{1em}
\Readsolutionfile{indications}
}

\newcommand{\correction}{
\newpage
\hrule
\begin{center}
{\Large \bf Correction}
\end{center}
\hrule
\vspace{1em}
\Readsolutionfile{solutions}
}




\begin{document}
\Opensolutionfile{indications}[_\jobname_hints]
\Opensolutionfile{solutions}[_\jobname_sol]


\title{Loi des cosinus (alias Al-Kashi)}
\author{Damien Mégy}
\maketitle

Catalogue d'exos sur Pythagore généralisé (=loi des cosinus, =Al Kashi) pour le Club Mathématique de Nancy. Rédaction en cours, ne pas diffuser.

Wikipedia : \url{https://fr.wikipedia.org/wiki/Loi_des_cosinus}

Mettre tout ce qu'il y a ici: \url{https://studymath.github.io/trigonometry/2017/02/02/the-law-of-cosines.html}

Mettre la superbe preuve sans mots de \url{https://www.cut-the-knot.org/Curriculum/Geometry/GeoGebra/CosineLawAB.shtml}



\begin{exo}
La preier de \url{https://geometry.ru/articles/Luis_Casey.pdf}

Deux cercles sont tangent à un troisième. On calcule la distance entre les points de contact d'une tangente commune aux deux cercles en fonction  des trois rayons, et de la distance entre les pts de contact avec le troisième cercle.

Ceci sert ensuite à prouver le théorème de Casey, avec Ptolémée.
\end{exo}


\begin{exo}[Formules de Mahavira via Al-Kashi]
Soit $ABCD$ un quadrilatère inscriptible. On note $a=AB$, $b=BC$, $c=CD$ et $d=DA$.
Montrer que la diagonale $BD$ est calculable en fonction des côtés grâce à la formule
\[ BD = \frac{(ab+cd)(ac+bd)}{ad+bc}.\]
\begin{hint}
Il s'agit de montrer que $BD(ad+bc) = (ab+cd)(ac+bd)$.
Pour cela, écrire  $BD$ de deux manières distinctes avec Al-Kashi. 
Quelle est la relation entre les angles $\widehat A$ et $\widehat C$ ?
\end{hint}
\begin{sol}
On a d'une part 
\[BD  = b^2+c^2-2bc\cos \widehat C,\]
et d'autre part 
\[ BD = a^2+d^2-2ad\cos\widehat A =a^2+d^2-2ad\cos\widehat C, \]
car $\widehat A = \pi - \widehat C$ puisque le quadrilatère est inscriptible.

On multiplie la première expression par $ad$ et la seconde par $bc$, on somme et on vérifie que ça se factorise comme annoncé.
\end{sol}
\end{exo}



\begin{exo}[Le théorème d'Appolonius / (premier) théorème de la médiane]
Soit $ABC$ un triangle de côtés $a$, $b$ et $c$, et soit $P$ le milieu du segment $[BC]$. On note $p=AP$.
Montrer que l'on a 
\[ 2(b^2+c^2)=a^2+4p^2.\]
\begin{hint}
La droite $(AP)$ coupe le cercle circonscrit en un point $D$. 
\end{hint}
\begin{sol}


Preuve avec la loi des cosinus : \url{https://en.wikipedia.org/wiki/Apollonius%27s_theorem}

Sinon, Ptolémée, mais lorsque placé dans l'autre feuille : 

On applique Ptolémée dans le quadrilatère $ABDC$ puis on utilise deux paires de triangles semblables pour écrire toutes les distances en fonction de $a$, $b$, $c$ et $p$.
\end{sol}
\end{exo}


\begin{exo}
L'identité du parallélogramme : \url{https://en.wikipedia.org/wiki/Parallelogram_law} preuve avec loi des cosinus.
\begin{hint}
\end{hint}
\begin{sol}
\end{sol}
\end{exo}



\begin{exo}[Deuxième théorème de la médiane]
Voir wiki
\begin{hint}
\end{hint}
\begin{sol}
\end{sol}
\end{exo}

\begin{exo}[Troisième théorème de la médiane]
Voir wiki
\begin{hint}
\end{hint}
\begin{sol}
\end{sol}
\end{exo}

\begin{exo}[Formule de Héron]
Voir wiki : \url{https://fr.wikipedia.org/wiki/Formule_de_H%C3%A9ron#D%C3%A9monstrations}
\begin{hint}
\end{hint}
\begin{sol}
\end{sol}
\end{exo}


\begin{exo}[Le théorème de Stewart]
Soit $ABC$ un triangle de côtés $a$, $b$ et $c$, et soit $P$ un point du segment $[BC]$. On note $p=AP$, $m=BP$ et $n=PC$.
Montrer que l'on a 
\[ b^2m+c^2,=a(p^2+mn)\]
\begin{hint}
La droite $(AP)$ coupe le cercle circonscrit en un point $D$. 
\end{hint}
\begin{sol}
On applique Ptolémée dans le quadrilatère $ABDC$ puis on utilise deux paires de triangles semblables pour écrire toutes les distances en fonction de $a$, $b$, $c$, $m$, $n$, $p$.

Il y a aussi une preuve avec la loi des cosinus, à mettre dans l'autre feuille.
\end{sol}
\end{exo}


\begin{exo}[Stewart pour les quadrilatères convexes]

\url{https://fr.wikipedia.org/wiki/Th%C3%A9or%C3%A8me_de_Stewart#Cas_du_quadrilat%C3%A8re_convexe}
Al-Kashi dans les quatre triangles.
\begin{hint}
\end{hint}
\begin{sol}
\end{sol}
\end{exo}


\begin{exo}
\begin{hint}
\end{hint}
\begin{sol}
\end{sol}
\end{exo}


\Closesolutionfile{indications}
\Closesolutionfile{solutions}

\indications
\correction

\end{document}

