\documentclass[11pt,a4paper]{article}
\usepackage[french]{babel}
\usepackage[utf8]{inputenc}
\usepackage{mathtools,amssymb,amsthm}
\usepackage{mathrsfs,stmaryrd}
\usepackage{fancybox,mdframed,multicol,comment,enumitem}
\usepackage{microtype}

\usepackage{hyperref}
\hypersetup{
    colorlinks=true,       % false: boxed links; true: colored links
    linkcolor=blue,          % color of internal links
    citecolor=blue,        % color of links to bibliography
    filecolor=blue,      % color of file links
    urlcolor=black           % color of external links
}

%\usepackage[dvipsnames]{xcolor}

\usepackage[normalem]{ulem} % pour souligner avec changements de ligne
\usepackage{pgf,pgfmath,tikz}
\usetikzlibrary{arrows}
\usetikzlibrary[patterns]
\tikzset{every picture/.style={execute at begin picture={
   \shorthandoff{:;!?};}
}}

\newcommand*\circled[1]{\tikz[baseline=(char.base)]{
            \node[shape=circle,draw,inner sep=2pt] (char) {#1};}}


\def\point{node {$\bullet$}}

\usepackage{tkz-euclide}



\theoremstyle{definition}
\newtheorem{theoreme}{Théorème}[section]
\newtheorem{definition}[theoreme]{Définition}
\newtheorem{definitions}[theoreme]{Définitions}
\newtheorem{lemme}[theoreme]{Lemme}
\newtheorem{proposition}[theoreme]{Proposition}
\newtheorem{corollaire}[theoreme]{Corollaire}
\newtheorem{remarque}[theoreme]{Remarque}
\newtheorem{ex}{Problème}

\newcommand{\N}{\mathbb N}
\newcommand{\Z}{\mathbb Z}
\newcommand{\Q}{\mathbb Q}
\newcommand{\R}{\mathbb R}
\newcommand{\C}{\mathbb C}
\newcommand{\U}{\mathbb U}
\newcommand{\F}{\mathbb F}
\newcommand{\G}{\mathbb G}

\newcommand*{\etoile}
{
\begin{center}
$\star$\par
$\star$\hspace*{3ex}$\star$
\end{center}
}

\newcommand{\ensemble}[2]{\left \{ #1  
    \ifx&#2&%
       %
    \else%
       \, \middle | \, #2%
    \fi%
\right \}}

\newcommand{\modulo}[1]{\:\left(\operatorname{mod}\:#1\right)}

% délimiteurs

\DeclarePairedDelimiter{\abs}{\lvert}{\rvert}
\DeclarePairedDelimiter{\ceil}{\lceil}{\rceil}
\DeclarePairedDelimiter{\floor}{\lfloor}{\rfloor}


%%%%%%%%%%%%%%%%%%%%%%%%%%%%%%%%%
%%%%%% MISE EN FORME CLUB %%%%%%%
%%%%%%%%%%%%%%%%%%%%%%%%%%%%%%%%%

\pagestyle{empty}

\usepackage[margin=2.5cm]{geometry}
\everymath{\displaystyle}
\usepackage{fourier}
%\usepackage{fourier,eulervm}% Adobe Utopia et Euler
% En-tête des feuilles :

\newcommand{\enTete}[1]{
\noindent \textbf{\textsf{\href{http://depmath-nancy.univ-lorraine.fr/club/}{Club Mathématique de Nancy} \hfill Institut Élie Cartan}}
\hrule
\begin{center}
{\Huge \textbf{#1}}
\end{center}
\hrule
\vspace{1em}
}

\newcommand{\avertissement}{\begin{mdframed}[linewidth=1pt]\textbf{AVERTISSEMENT ! Ce document est un brouillon qui sert de catalogue pour les feuilles d'exos du club mathématique de Nancy \url{https://dmegy.perso.math.cnrs.fr/club/}. Ne pas diffuser tel quel aux élèves ni de façon large sur le net, il reste des coquilles et énoncés parfois peu précis. Ce document a vocation a rester inachevé. Il peut néanmoins être utile aux enseignants. Enfin, ce document change en permanence, la version à jour est récupérable sur \url{https://github.com/dmegy/clubmath-exos}.}\end{mdframed}}




% - - - - - - - - - - - - - -
% PARAMETRAGE DU PACKAGE ANSWERS 
% POUR LES INDICATIONS ET CORRECTIONS
% - - - - - - - - - - - - - - 

\usepackage{answers}

\Newassociation{sol}{Soln}{solutions}
% ira dans le fichier d'identifiant 'solutions'
% et écrira les solutions dans un environnement 'Soln'
\Newassociation{hint}{Hint}{indications}

\newenvironment{exo}{\begin{ex} \label{enonce.\theex} }{\end{ex} }

\renewenvironment{Soln}[1]{\noindent{\bf Correction de l'exercice \ref{enonce.#1}.} \\ }

\renewenvironment{Hint}[1]{ \noindent{\bf Exercice  \ref{enonce.#1}.} \label{hint.#1}}


% - - - - - - - - - - - - - - 
% FIN PARAMETRAGE ANSWERS
% - - - - - - - - - - - - - - 

%-----------------------------
% MACROS POUR LES FEUILLES DE TD


\newenvironment{feuilleTD}{


\Opensolutionfile{indications}[\jobname_hints]
\Opensolutionfile{solutions}[\jobname_sol]
}{
\Closesolutionfile{indications}
\Closesolutionfile{solutions}
}

\newcommand{\indications}{
\newpage
\noindent {\Large \bf Indications} \hrulefill

\vspace{1em}
\Readsolutionfile{indications}
}

\newcommand{\correction}{
\newpage
\hrule
\begin{center}
{\Large \bf Correction}
\end{center}
\hrule
\vspace{1em}
\Readsolutionfile{solutions}
}




\begin{document}
\Opensolutionfile{indications}[_\jobname_hints]
\Opensolutionfile{solutions}[_\jobname_sol]


\title{Loi des sinus}
\author{Damien Mégy}
\maketitle

Prérequis : Pythagore, Angle inscrit/angle au centre, Thalès, parfois Al-Kashi.

\section{Sans Al-Kashi}

\begin{exo}[Une preuve de la loi des sinus]
Soit $ABC$ un triangle, dont on note $a$, $b$ et $c$ les côtés et $\widehat A$, $\widehat B$ et $\widehat C$ les angles.
On note $O$ le centre du cercle  circonscrit et $R$ son rayon.
Montrer que $\dfrac{a}{\sin \widehat A} = 2R$.
En déduire que
\[ \frac{a}{\sin \widehat A} = \frac{b}{\sin \widehat B}  = \frac{c}{\sin \widehat C}.\]
C'est la \emph{loi des sinus}.
\begin{hint}
Projeter $O$ sur $[BC]$, et utiliser le théorème de l'angle au centre.
\end{hint}
\begin{sol}
Soit $P$ le projeté orthogonal de $O$ sur $[BC]$.
On a $\widehat{BOC} = 2\widehat{A}$, et donc $\widehat{BOP}=\widehat A$.
On en déduit 
\[ \sin \widehat A = \sin \widehat{BOP}
= \frac{BP}{BO} = \frac{a}{2R}.\]
\end{sol}
\end{exo}


\begin{exo}[Une autre preuve de la loi des sinus]
Celle de l'autre feuille d'exos : on se ramène à un triangle rectangle
\begin{hint}
\end{hint}
\begin{sol}
\end{sol}
\end{exo}

\begin{exo}[Une troisième  preuve de la loi des sinus (aires) ?]
\begin{hint}
\end{hint}
\begin{sol}
\end{sol}
\end{exo}

\begin{exo}[Application directe]
Soit $ABC$ un triangle de côtés $a$, $b$ et $c$. Montrer que
\[ \frac{a+b}{c}=\frac{\sin \widehat A+\sin \widehat B}{\sin \widehat C}\]
\begin{hint}
Faire apparaître le facteur de proportionnalité.
\end{hint}
\begin{sol}
\end{sol}
\end{exo}


\begin{exo}
Soit $ABC$ un triangle, $a$, $b$ et $c$ ses côtés et $R$ le rayon de son cercle circonscrit.
Montrer que son aire est égale à $\dfrac{abc}{4R}$.
\begin{hint}
\end{hint}
\begin{sol}
On applique juste la loi des sinus.
\end{sol}
\end{exo}

\begin{exo}
\begin{hint}
\end{hint}
\begin{sol}
\end{sol}
\end{exo}

\begin{exo}
\begin{hint}
\end{hint}
\begin{sol}
\end{sol}
\end{exo}


\section{Avec Al-Kashi}

\begin{exo}
\begin{hint}
\end{hint}
\begin{sol}
\end{sol}
\end{exo}


\Closesolutionfile{indications}
\Closesolutionfile{solutions}

\indications
\correction

\end{document}

