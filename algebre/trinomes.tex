\documentclass[11pt,a4paper]{article}
\usepackage[french]{babel}
\usepackage[utf8]{inputenc}
\usepackage{mathtools,amssymb,amsthm}
\usepackage{mathrsfs,stmaryrd}
\usepackage{fancybox,mdframed,multicol,comment,enumitem}
\usepackage{microtype}

\usepackage{hyperref}
\hypersetup{
    colorlinks=true,       % false: boxed links; true: colored links
    linkcolor=blue,          % color of internal links
    citecolor=blue,        % color of links to bibliography
    filecolor=blue,      % color of file links
    urlcolor=black           % color of external links
}

%\usepackage[dvipsnames]{xcolor}

\usepackage[normalem]{ulem} % pour souligner avec changements de ligne
\usepackage{pgf,pgfmath,tikz}
\usetikzlibrary{arrows}
\usetikzlibrary[patterns]
\tikzset{every picture/.style={execute at begin picture={
   \shorthandoff{:;!?};}
}}

\newcommand*\circled[1]{\tikz[baseline=(char.base)]{
            \node[shape=circle,draw,inner sep=2pt] (char) {#1};}}


\def\point{node {$\bullet$}}

\usepackage{tkz-euclide}



\theoremstyle{definition}
\newtheorem{theoreme}{Théorème}[section]
\newtheorem{definition}[theoreme]{Définition}
\newtheorem{definitions}[theoreme]{Définitions}
\newtheorem{lemme}[theoreme]{Lemme}
\newtheorem{proposition}[theoreme]{Proposition}
\newtheorem{corollaire}[theoreme]{Corollaire}
\newtheorem{remarque}[theoreme]{Remarque}
\newtheorem{ex}{Problème}

\newcommand{\N}{\mathbb N}
\newcommand{\Z}{\mathbb Z}
\newcommand{\Q}{\mathbb Q}
\newcommand{\R}{\mathbb R}
\newcommand{\C}{\mathbb C}
\newcommand{\U}{\mathbb U}
\newcommand{\F}{\mathbb F}
\newcommand{\G}{\mathbb G}

\newcommand*{\etoile}
{
\begin{center}
$\star$\par
$\star$\hspace*{3ex}$\star$
\end{center}
}

\newcommand{\ensemble}[2]{\left \{ #1  
    \ifx&#2&%
       %
    \else%
       \, \middle | \, #2%
    \fi%
\right \}}

\newcommand{\modulo}[1]{\:\left(\operatorname{mod}\:#1\right)}

% délimiteurs

\DeclarePairedDelimiter{\abs}{\lvert}{\rvert}
\DeclarePairedDelimiter{\ceil}{\lceil}{\rceil}
\DeclarePairedDelimiter{\floor}{\lfloor}{\rfloor}


%%%%%%%%%%%%%%%%%%%%%%%%%%%%%%%%%
%%%%%% MISE EN FORME CLUB %%%%%%%
%%%%%%%%%%%%%%%%%%%%%%%%%%%%%%%%%

\pagestyle{empty}

\usepackage[margin=2.5cm]{geometry}
\everymath{\displaystyle}
\usepackage{fourier}
%\usepackage{fourier,eulervm}% Adobe Utopia et Euler
% En-tête des feuilles :

\newcommand{\enTete}[1]{
\noindent \textbf{\textsf{\href{http://depmath-nancy.univ-lorraine.fr/club/}{Club Mathématique de Nancy} \hfill Institut Élie Cartan}}
\hrule
\begin{center}
{\Huge \textbf{#1}}
\end{center}
\hrule
\vspace{1em}
}

\newcommand{\avertissement}{\begin{mdframed}[linewidth=1pt]\textbf{AVERTISSEMENT ! Ce document est un brouillon qui sert de catalogue pour les feuilles d'exos du club mathématique de Nancy \url{https://dmegy.perso.math.cnrs.fr/club/}. Ne pas diffuser tel quel aux élèves ni de façon large sur le net, il reste des coquilles et énoncés parfois peu précis. Ce document a vocation a rester inachevé. Il peut néanmoins être utile aux enseignants. Enfin, ce document change en permanence, la version à jour est récupérable sur \url{https://github.com/dmegy/clubmath-exos}.}\end{mdframed}}




% - - - - - - - - - - - - - -
% PARAMETRAGE DU PACKAGE ANSWERS 
% POUR LES INDICATIONS ET CORRECTIONS
% - - - - - - - - - - - - - - 

\usepackage{answers}

\Newassociation{sol}{Soln}{solutions}
% ira dans le fichier d'identifiant 'solutions'
% et écrira les solutions dans un environnement 'Soln'
\Newassociation{hint}{Hint}{indications}

\newenvironment{exo}{\begin{ex} \label{enonce.\theex} }{\end{ex} }

\renewenvironment{Soln}[1]{\noindent{\bf Correction de l'exercice \ref{enonce.#1}.} \\ }

\renewenvironment{Hint}[1]{ \noindent{\bf Exercice  \ref{enonce.#1}.} \label{hint.#1}}


% - - - - - - - - - - - - - - 
% FIN PARAMETRAGE ANSWERS
% - - - - - - - - - - - - - - 

%-----------------------------
% MACROS POUR LES FEUILLES DE TD


\newenvironment{feuilleTD}{


\Opensolutionfile{indications}[\jobname_hints]
\Opensolutionfile{solutions}[\jobname_sol]
}{
\Closesolutionfile{indications}
\Closesolutionfile{solutions}
}

\newcommand{\indications}{
\newpage
\noindent {\Large \bf Indications} \hrulefill

\vspace{1em}
\Readsolutionfile{indications}
}

\newcommand{\correction}{
\newpage
\hrule
\begin{center}
{\Large \bf Correction}
\end{center}
\hrule
\vspace{1em}
\Readsolutionfile{solutions}
}



\begin{document}
\Opensolutionfile{indications}[_\jobname_hints]
\Opensolutionfile{solutions}[_\jobname_sol]


\title{Trinômes}
\author{Damien Mégy}
\maketitle


\section{Compléter le carré}

Feuille entière.

Google "complete the square" 

\section{Factorisation}


Rappel :  $(a+b)^2=a^2+2ab+b^2$, $(a-b)^2=a^2-2ab+b^2$ et $(a+b)(a-b)=a^2-b^2$.

\begin{exo}
Factoriser les expression suivantes : 
\[ a^2+4a+4, \quad
b^2+20b+100,\quad
c^2+14c+49,\quad
d^2+26d+169,\quad
e^2+22e+121,\quad
f^2+30f+225,
\]
\[ 
4a^2+4a+1,\quad
9b^2+6b+1,\quad
64c^2+16c+1,\quad
49d^2+14d+1,\quad
144e^2+24e+1,\quad
256f^2+32f+1,
\]
\[
4a^2+8a+4,\quad
4b^2+12b+9,\quad
25c^2+20c+4,\quad
9d^2+12d+4,\quad
9e^2+30e+25,\quad
36f^2+108f+81.
\]
\begin{hint}
Pour les premiers de chaque ligne : $(a+2)^2$, $(2a+1)^2$ et $(2a+2)^2$. À noter que pour ce dernier trinôme, on peut commencer par factoriser l'expression par $4$ ce qui donne $4(a^2+2a+1)$, puis utiliser l'identité remarquable : $4(a+1)^2$, puis \og rentrer\fg{} le $4$ dans le carré en écrivant $4(a+1)^2 = 2^2(a+1)^2 = \big(2(a+1)\big)^2 = (2a+2)^2$.
\end{hint}
\end{exo}

\begin{exo}
Factoriser les expression suivantes : 
\[ 
x^2+2x\sqrt{5}+5,\quad
y^2+y\sqrt{20}+5,\quad
z^2+z\sqrt{12}+3,\quad
t^2+4t\sqrt{3}+12,\quad
u^2+6u\sqrt{2}+18,
\]
\[
3a^2+2a\sqrt{3}+1,\quad
7b^2+2b\sqrt{7}+1,\quad
12c^2+4c\sqrt{3}+1,\quad
18d^2+6d\sqrt{2}+1,\quad
72e^2+12e\sqrt{2}+1.
\]
\end{exo}


\begin{exo}
Factoriser les expressions suivantes :
\[
x^2-4,\quad 
y^2-144,\quad 
z^2-169,\quad 
t^2-225,\quad 
4u^2-121,\quad
9v^2-81,\quad
36w^2-25,
\]
\[
x^2-5,\quad
x^2-20,\quad
3x^2-1,\quad
12x^2-1,\quad
81x^2-7,\quad
7x^2-121,\quad
145x^2-144,\quad
11x^2-40.
\]
\end{exo}



\begin{exo}
Si on se donne deux nombres $S$ (pour \og somme \fg) et $P$ (pour \og produit\fg), il est parfois possible de deviner deux nombres $a$ et $b$ tels que $a+b=S$ et $ab=P$. Par exemple, si on donne $S=6$ et $P=5$, on peut deviner $a=5$ et $b=1$ (ou l'inverse) : on a alors bien $a+b = 6 = S$ et $ab = 5 = P$. De même, si on se donne $S=5$ et $P=6$, on peut trouver de tête que $S=5=2+3$ et $P=6 = 2\times 3$. Savoir faire ce type de calcul de tête est extrêmement utile. 
Pour chaque couple d'entiers $S$ et $P$, déterminer des nombres entiers $a$ et $b$ tels que $a+b=S$ et $ab=P$. Attention, dans la troisième ligne, les nombres $a$ et $b$ peuvent être négatifs.
\[
S=3 \text{ et } P=2, \quad
S=6\text{ et } P=8, \quad
S=5\text{ et } P=4, \quad
S=13\text{ et } P=12, \quad
S=12\text{ et } P=35,\quad
\]
\[
S=13\text{ et } P=36,\quad
S=20\text{ et } P=36,\quad
S=37\text{ et } P=36,\quad
S=15\text{ et } P=36,\quad
S=12\text{ et } P=36,\quad
\]
\[
S=1 \text{ et } P=-2, \quad
S=1 \text{ et } P=-6, \quad
S=5 \text{ et } P=-6, \quad
S=-6 \text{ et } P=5, \quad
S=-5 \text{ et } P=6. \quad
\]
\end{exo}

\begin{exo}[Échauffement pour la suite]
Développer et réduire les expressions suivantes : 
\[ (x+1)(x+7), \quad
(x+3)(x+4), \quad
(x+11)(x+2), \quad
(x+12)(x+5), \quad
(2x+1)(x+3), \quad
(3x+1)(2x+5).
\]
\end{exo}

\begin{exo}
Si on cherche à factoriser le trinôme $x^2+5x+6$, on \og voit\fg{} (avec de l'entraînement) que $5=2+3$ et $6=2\times 3$ et donc on peut trouver rapidement la factorisation $x^2+5x+6 = (x+2)(x+3)$.
Utiliser ce principe pour factoriser les trinômes suivants:
\[
x^2+3x+2,\quad
x^2+7x+6,\quad
x^2+13x+12,\quad
x^2+7x+12,\quad
x^2+6x+8,\quad
x^2+8x+12,\quad
x^2+12x+35,
\]
\[
x^2+x-2,\quad
x^2+x-6,\quad
x^2-x-20,\quad
x^2-5x+6,\quad
x^2-3x+2,\quad
x^2-5x+4,\quad
x^2-6x+8,\quad
x^2-7x+10.
\]

\end{exo}


\section{Conditions sur le graphe}

Le graphe est ...




\begin{exo}
Déterminer les couples $(b,c)$ de réels tels que le graphe du trinôme $x^2+bx+c$ passe par les points de coordonnées $(1,1)$ et $(3,0)$.
% système 2x2
\begin{hint}
\end{hint}
\begin{sol}
\end{sol}
\end{exo}


\begin{exo}
Déterminer les triplets $(a,b,c)$ de nombres réels tels que le graphe du trinôme $ax^2+bx+c$ passe par les points de coordonnées $(2,1)$, $(3,2)$ et $(5,0)$.

\begin{hint}
\end{hint}
\begin{sol}
On trouve un système $2\times 3$ dont la résolution donne $a=-2/3$, $b=13/3$ et $c=-5$.
\end{sol}
\end{exo}

\begin{exo}
Soient $P$ et $Q$ les points du plan de coordonnées  $(2,3)$ et $(4,5)$.
Parmi les trinômes $ax^2+bx+c$ dont le graphe passe par les deux points $P$ et $Q$, lesquels ont un graphe qui passe par l'origine ?
\begin{hint}
\end{hint}
\begin{sol}
\end{sol}
\end{exo}

\begin{exo}
Soient $P$ et $Q$ les points du plan de coordonnées  $(2,3)$ et $(4,0)$.
Parmi les trinômes $ax^2+bx+c$ dont le graphe passe par les deux points $P$ et $Q$, lesquels ont un graphe qui est tangent à l'axe des abscisses ?
\begin{hint}
\end{hint}
\begin{sol}
\end{sol}
\end{exo}


\begin{exo}
Déterminer les trinômes $ax^2+bx+c$ dont le graphe admet l'axe de symétrie $x=3$ et contient les points de coordonnées $(2,2)$ et $(5,-1)$.
\begin{hint}
Distinguer suivant si $a$ est nul ou pas.
\end{hint}
\begin{sol}
Si $a\neq 0$, on trouve $a=-1$ et $b=6$ et $c=-6$. (Si $a=0$ il n'y a pas de solution.)
\end{sol}
\end{exo}

\begin{exo}
Soit $p$ un nombre réel et considérons le trinôme $3x^2+px+3$.
Sachant qu'une de ses racines est le carré de l'autre, que vaut $p$ ?
\begin{hint}
\end{hint}
\begin{sol}
$-6$.
\end{sol}
\end{exo}

\begin{exo}
Déterminer le nombre réel $b$ de sorte que le trinôme $x^2+bx+3$ possède deux racines à distance un l'une de l'autre, c'est-à-dire deux racines $\alpha$ et $\beta$ telles que $|\alpha-\beta|=1$.
\begin{hint}
\end{hint}
\begin{sol}
Discriminant, ou bien formules de Viète.
\end{sol}
\end{exo}

\begin{exo}
Soient $a<b$ deux nombres réels.
Combien de racines le polynôme $(x-a)(x-b)-1$ possède-il dans les intervalles $]-\infty, a[$,  $]a,b[$ et  $]b,+\infty[$ ?
\begin{hint}
\end{hint}
\begin{sol}
\end{sol}
\end{exo}



\begin{exo}
\begin{hint}
\end{hint}
\begin{sol}
\end{sol}
\end{exo}



\begin{exo}
\begin{hint}
\end{hint}
\begin{sol}
\end{sol}
\end{exo}



\begin{exo}
\begin{hint}
\end{hint}
\begin{sol}
\end{sol}
\end{exo}



\begin{exo}
\begin{hint}
\end{hint}
\begin{sol}
\end{sol}
\end{exo}

\section{La parabole $y=x^2$}

Exo sur le foyer, la définition avec foyer et directrice, cercles osculateurs...

Paraboles orthogonales, paraboles tangentes.

Tous les exercices de \url{https://www.cbsd.org/cms/lib/PA01916442/Centricity/Domain/2780/333202_1002_735-743.pdf}

exos de \url{https://fr.scribd.com/document/402953420/9-PARABOLA-EXERCISE-pdf}

\begin{exo}
\begin{hint}
\end{hint}
\begin{sol}
\end{sol}
\end{exo}

\begin{exo}
\begin{hint}
\end{hint}
\begin{sol}
\end{sol}
\end{exo}

\begin{exo}
\begin{hint}
\end{hint}
\begin{sol}
\end{sol}
\end{exo}

\Closesolutionfile{indications}
\Closesolutionfile{solutions}

\indications
\correction

\end{document}

